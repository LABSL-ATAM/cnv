\documentclass[]{article}
\usepackage{lmodern}
\usepackage{amssymb,amsmath}
\usepackage{ifxetex,ifluatex}
\usepackage{fixltx2e} % provides \textsubscript
\ifnum 0\ifxetex 1\fi\ifluatex 1\fi=0 % if pdftex
  \usepackage[T1]{fontenc}
  \usepackage[utf8]{inputenc}
\else % if luatex or xelatex
  \ifxetex
    \usepackage{mathspec}
    \usepackage{xltxtra,xunicode}
  \else
    \usepackage{fontspec}
  \fi
  \defaultfontfeatures{Mapping=tex-text,Scale=MatchLowercase}
  \newcommand{\euro}{€}
\fi
% use upquote if available, for straight quotes in verbatim environments
\IfFileExists{upquote.sty}{\usepackage{upquote}}{}
% use microtype if available
\IfFileExists{microtype.sty}{%
\usepackage{microtype}
\UseMicrotypeSet[protrusion]{basicmath} % disable protrusion for tt fonts
}{}
\ifxetex
  \usepackage[setpagesize=false, % page size defined by xetex
              unicode=false, % unicode breaks when used with xetex
              xetex]{hyperref}
\else
  \usepackage[unicode=true]{hyperref}
\fi
\usepackage[usenames,dvipsnames]{color}
\hypersetup{breaklinks=true,
            bookmarks=true,
            pdfauthor={},
            pdftitle={},
            colorlinks=true,
            citecolor=blue,
            urlcolor=blue,
            linkcolor=magenta,
            pdfborder={0 0 0}}
\urlstyle{same}  % don't use monospace font for urls
\usepackage{longtable,booktabs}
\usepackage{graphicx,grffile}
\makeatletter
\def\maxwidth{\ifdim\Gin@nat@width>\linewidth\linewidth\else\Gin@nat@width\fi}
\def\maxheight{\ifdim\Gin@nat@height>\textheight\textheight\else\Gin@nat@height\fi}
\makeatother
% Scale images if necessary, so that they will not overflow the page
% margins by default, and it is still possible to overwrite the defaults
% using explicit options in \includegraphics[width, height, ...]{}
\setkeys{Gin}{width=\maxwidth,height=\maxheight,keepaspectratio}
\setlength{\parindent}{0pt}
\setlength{\parskip}{6pt plus 2pt minus 1pt}
\setlength{\emergencystretch}{3em}  % prevent overfull lines
\providecommand{\tightlist}{%
  \setlength{\itemsep}{0pt}\setlength{\parskip}{0pt}}
\setcounter{secnumdepth}{0}

\date{}

% Redefines (sub)paragraphs to behave more like sections
\ifx\paragraph\undefined\else
\let\oldparagraph\paragraph
\renewcommand{\paragraph}[1]{\oldparagraph{#1}\mbox{}}
\fi
\ifx\subparagraph\undefined\else
\let\oldsubparagraph\subparagraph
\renewcommand{\subparagraph}[1]{\oldsubparagraph{#1}\mbox{}}
\fi

\begin{document}

\section{Convenio Colectivo Para El Sector No Docente De Las
Instituciones Universitarias
Nacionales}\label{convenio-colectivo-para-el-sector-no-docente-de-las-instituciones-universitarias-nacionales}

\hyperref[principios-generales]{PRINCIPIOS GENERALES}\\
\hyperref[condiciones-para-el-ingreso-y-egreso]{CONDICIONES INGRESO Y
EGRESO}\\
\hyperref[regimen-de-concursos]{REGIMEN DE CONCURSOS}\\
\hyperref[retribuciones]{RETRIBUCIONES}\\
\hyperref[tiempo-de-trabajo]{TIEMPO DE TRABAJO}\\
\hyperref[salud-e-higiene]{SALUD E HIGIENE}\\
\hyperref[capacitacion]{CAPACITACIÓN}\\
\hyperref[evaluacion-de-desempeuxd1o]{EVALUACIÓN DE DESEMPEÑO}\\
\hyperref[regimen-disciplinario]{DISCIPLINA}

\textbf{UNIVERSIDADES NACIONALES}

\textbf{Decreto 366/2006}

\textbf{Homológase el Convenio Colectivo de Trabajo para el Sector No
Docente de las Instituciones Universitarias Nacionales celebrado por el
Consejo Interuniversitario Nacional y la Federación Argentina de
Trabajadores de las Universidades Nacionales, de fecha 16 de junio de
2005.}

Bs. As., 31/3/2006

VISTO el Expediente Nº 948.S013/93 del Registro del entonces MINISTERIO
DE TRABAJO Y SEGURIDAD SOCIAL, las Leyes Nros. 11.672 (t.o. 2005)
Complementaria Permanente de Presupuesto, 24.447 y 24.938, el Decreto
Nro. 1007 de fecha 7 de julio de 1995, y

CONSIDERANDO:

Que por imperio de los artículos 22 y 23 de la Ley Nº 24.938 se creó el
PROGRAMA DE REFORMA Y REESTRUCTURACION LABORAL en el ámbito de la
Universidades Nacionales.

Que las modalidades y condiciones de dicho PROGRAMA DE REFORMA Y
REESTRUCTURACION LABORAL deben ser acordadas mediante negociaciones
colectivas, tal como lo impone la normativa nacional vigente, en un todo
de acuerdo con las disposiciones resultantes del Convenio Nº 154 de
fomento de la Negociación Colectiva de la ORGANIZACIÓN INTERNACIONAL DEL
TRABAJO (O.I.T.), ratificado mediante Ley Nº 23.544.

Que de conformidad con lo dispuesto en el artículo 119 de la Ley Nº
11.672 (t.o. 2005) Complementaria Permanente de Presupuesto, ``El
dictado del acto administrativo que ponga en vigencia los acuerdos a los
que se arribe en las respectivas Comisiones Negociadoras estará
condicionado al cumplimiento de las pautas y mecanismos que contemplen
la revisión de los regímenes de obligaciones docentes, de antigüedad y
de incompatibilidades en el caso del personal docente y la mayor
productividad, capacitación y contracción a las tareas en el caso del
personal no docente'' de las Universidades.

Que a fojas 593/630 del Expediente citado en el Visto, obra el texto
ordenado del Convenio Colectivo de Trabajo para el Sector No Docente de
las Instituciones Universitarias Nacionales celebrado entre el CONSEJO
INTERUNIVERSITARIO NACIONAL (C.I.N) y la FEDERACION ARGENTINA DE
TRABAJADORES DE LAS UNIVERSIDADES NACIONALES (F.A.T.U.N.), que prevé,
los derechos y obligaciones de las partes, fija las condiciones para el
ingreso y egreso, el régimen de concursos, tiempo de trabajo, pautas de
salud e higiene, capacitación, evaluación de desempeño y régimen
disciplinario.

Que las representaciones invocadas por las partes celebrantes han sido
fehacientemente acreditadas, habiéndose integrado la Comisión
Negociadora de conformidad con lo dispuesto por el artículo 2º del
Decreto Nº 1007/95.

Que se ha otorgado al MINISTERIO DE EDUCACION, CIENCIA Y TECNOLOGIA DE
LA NACION la intervención que prevé el artículo 22 de la Ley Nº 24.938.

Que en consecuencia corresponde homologar el acuerdo obrante a fojas
593/630 del Expediente citado en el Visto, cuya copia autenticada se
incorpora como Anexo del presente Decreto.

Que la Dirección General de Asuntos Jurídicos del MINISTERIO DE TRABAJO,
EMPLEO Y SEGURIDAD SOCIAL ha tomado la intervención que le compete.

Que el presente acto se dicta en función de las atribuciones conferidas
por el artículo 99, inciso 1, de la CONSTITUCION NACIONAL y el Decreto
Nº 1007 de fecha 7 de julio de 1995.

Por ello,

EL PRESIDENTE DE LA NACION ARGENTINA

DECRETA:

\begin{center}\rule{0.5\linewidth}{\linethickness}\end{center}

Artículo 1º --- Homológase el Convenio Colectivo de Trabajo para el
Sector No Docente de las Instituciones Universitarias Nacionales,
celebrado por el CONSEJO INTERUNIVERSITARIO NACIONAL (C.I.N) y la
FEDERACION ARGENTINA DE TRABAJADORES DE LAS UNIVERSIDADES NACIONALES
(F.A.T.U.N.), sector gremial que representa al Personal No Docente de
las Instituciones Universitarias Nacionales, de fecha 16 de junio de
2005, cuya copia autenticada se incorpora como Anexo del presente
Decreto.

\begin{center}\rule{0.5\linewidth}{\linethickness}\end{center}

Art. 2º --- Comuníquese, publíquese, dése a la Dirección Nacional del
Registro Oficial y archívese. --- KIRCHNER. --- Daniel F. Filmus. ---
Carlos A. Tomada.

\section{CONVENIO COLECTIVO PARA EL SECTOR NO DOCENTE DE LAS
INSTITUCIONES UNIVERSITARIAS
NACIONALES}\label{convenio-colectivo-para-el-sector-no-docente-de-las-instituciones-universitarias-nacionales-1}

\subsection{TÍTULO 1}\label{tuxedtulo-1}

\subsubsection{Partes contratantes y acreditación de
personería}\label{partes-contratantes-y-acreditaciuxf3n-de-personeruxeda}

Art. 1º: La FATUN, con personería gremial Nº 1394/74, con domicilio en
Medrano 843, 1º piso de la Ciudad Autónoma de Buenos Aires, en adelante
la PARTE TRABAJADORA; y las instituciones universitarias nacionales, las
que oportunamente unificaron su personería como parte empleadora,
constituyendo domicilio a estos efectos en Pacheco de Melo 2084, Ciudad
Autónoma de Buenos Aires, en adelante LA EMPLEADORA, convienen en
celebrar el presente convenio colectivo, de acuerdo a las Leyes números
24.185, 24.447 y 24.521, decreto reglamentario 1007/95 y Acuerdo
Plenario del Consejo Interuniversitario Nacional (CIN) número 182/95 y
estatuto de FATUN. Ambas partes acreditan su personería con la
documentación que se adjunta.

\subsubsection{Actividad y trabajadores a que se
refiere}\label{actividad-y-trabajadores-a-que-se-refiere}

Art. 2º: La presente Convención Colectiva de Trabajo comprende a todos
los trabajadores de las Instituciones Universitarias nacionales,
cualquiera sea su situación de revista, excluido el personal de
conducción política y los trabajadores docentes.

\subsubsection{Ambito de aplicación}\label{ambito-de-aplicaciuxf3n}

Art. 3º: El presente convenio será de aplicación en todo territorio
donde las Instituciones Universitarias nacionales tengan actividades de
cualquier tipo que sea, con las limitaciones sobre extraterritorialidad
que impongan las normas de Derecho laboral argentino.

\subsubsection{Período de vigencia}\label{peruxedodo-de-vigencia}

Art. 4º: El presente convenio colectivo tendrá una vigencia de dos años,
a contar desde el día siguiente a la publicación del decreto que lo
homologue. Las condiciones generales de trabajo y económicas
establecidas en esta convención colectiva regirán a partir del momento
acordado por las partes en cada caso. Las condiciones económicas podrán
ser revisadas a pedido de cualquiera de las partes, para analizar
circunstancias sobrevinientes que consideren relevantes, y especialmente
cuando corresponda en virtud de lo establecido por el art. 13º del
Decreto Nº 1007/95 o por el plan plurianual del Programa de Reforma y
Reestructuración Laboral oportunamente acordado, en caso de haberse
asignado nuevos fondos a ese Programa u otro similar que en adelante se
otorgue.

Art. 5º: Las partes acuerdan comenzar a negociar la nueva Convención
Colectiva o su renovación por lo menos tres meses antes de la fecha de
finalización de su vigencia. Si en ese lapso no se llegara a un acuerdo,
la Convención permanecerá vigente, salvo disposición de carácter general
en contrario, o que se den las circunstancias previstas en el artículo
anterior para su renegociación, lo que en ningún caso habilitará la
aplicación de la Ley de Contrato de Trabajo, quedando reservada la
calidad de empleo público autorregulado, en virtud de la capacidad de
las instituciones universitarias sobre la administración de su régimen
del personal, consagrado por la Ley de Educación Superior. Al respecto
las partes se obligan a negociar de buena fe, concurriendo a las
reuniones y audiencias concertadas en debida forma, designando
negociadores con el mandato correspondiente y aportando los elementos
para una discusión fundada, todo ello para alcanzar un acuerdo justo,
con resguardo de los mecanismos propios de adopción de las decisiones
regidos por el Acuerdo Plenario del CIN Nº 182/95 y el Estatuto de
FATUN.

\subsection{TÍTULO 2}\label{tuxedtulo-2}

\hyperdef{}{principios-generales}{\subsubsection{PRINCIPIOS
GENERALES}\label{principios-generales}}

Art. 6º: Fines compartidos: Constituye objeto esencial en el accionar de
las partes realizar las acciones tendientes a brindar el más eficaz
servicio en lo que a la actividad no docente corresponde.

En este sentido y ante la necesidad de adecuarse a los cambios que se
vienen produciendo en las Instituciones Universitarias nacionales, las
partes manifiestan su convicción de acordar y consensuar la
implementación de acciones coherentes para encontrar soluciones técnicas
y profesionales acordes. Convienen organizar las actividades de acuerdo
a las nuevas tecnologías, técnicas y equipamientos, que permitan hacer
más productivas las tareas y funciones del personal no docente,
utilizando la capacitación, los conocimientos y las habilidades de cada
uno y del conjunto de los trabajadores no docentes, los que a su vez se
prestarán a la capacitación en su actividad actual o la que
potencialmente resulte de sus nuevas habilidades, para el mejor
aprovechamiento de las nuevas formas de relación laboral acordadas, y
con especial atención a los objetivos institucionales.

Art. 7º: Prohibición de discriminación y deber de igualdad de trato: Se
prohibe cualquier tipo de discriminación entre los trabajadores de las
instituciones universitarias nacionales por motivos de raza, sexo,
religión, nacionalidad, políticos, gremiales o de edad. El empleador
debe dispensar a todos los trabajadores igual trato en identidad de
situaciones.

Art. 8º: Facultad de dirección: Quien tenga personal a cargo tiene
facultades para organizar técnicamente el trabajo de los agentes bajo su
responsabilidad, lo que incluye la facultad de dirección, que deberá
ejercitarse con carácter funcional, atendiendo a los objetivos de la
dependencia, y tomando en cuenta la preservación de los derechos del
trabajador.

Art. 9º: Para mantener el buen funcionamiento de las Instituciones
Universitarias Nacionales y la armonía de las relaciones laborales entre
las partes, la empleadora reconoce a la parte sindical signataria, en
todos sus niveles orgánicos, tanto a nivel general como de sus
organizaciones adheridas, como legítima representante de los
trabajadores, de acuerdo a la legislación vigente y en el marco de esta
negociación, asegurando la mejor convergencia posible de los puntos de
vista e intereses de las partes. Las Instituciones Universitarias se
comprometen a mantener informadas a las organizaciones representantes de
los trabajadores de aquellas medidas o decisiones que por su particular
importancia afecten sustancialmente los intereses de éstos, procurando
consensuarlas; a su vez los representantes de los trabajadores se
comprometen a transmitir esta información a sus representados, de manera
oportuna y veraz.

Art. 10º: Régimen de publicidad: Toda modificación al régimen de la
relación de empleo o del horario que se aplique individualmente al
personal no docente deberá ser notificado en forma escrita y fehaciente,
con copia al destinatario. En caso de modificaciones de carácter general
serán notificadas a través de los medios suficientemente idóneos que la
Institución Universitaria determine. El sindicato podrá utilizar las
carteleras o cualquier otro medio de comunicación colectiva, acordados
con el empleador.

Art. 11º: Del trabajador: El personal no docente permanente de las
Instituciones Universitarias nacionales tendrá los siguientes derechos:

a) Estabilidad.

b) Retribución por sus servicios.

c) Igualdad de oportunidades en la carrera.

d) Capacitación permanente.

e) Libre agremiación y negociación colectiva.

f) Licencias, justificaciones y franquicias.

g) Renuncia.

h) Jubilación o retiro.

i) Condiciones adecuadas que aseguren la higiene y seguridad en el
trabajo.

j) Derecho a la información de conformidad con lo establecido por la
Recomendación número 163 de la OIT.

k) Asistencia Social para sí y su familia, de acuerdo a la legislación
vigente.

I) Compensaciones e indemnizaciones.

m) Interposición de recursos.

n) Participación por intermedio de las Organizaciones Gremiales, de
acuerdo al presente Convenio.

La presente enumeración no es taxativa, y se enuncia sin perjuicio de
los acuerdos paritarios locales.

Art. 12º: Sin perjuicio de los deberes que en función de las
particularidades de la actividad desempeñada pudieran agregarse en los
respectivos convenios particulares, todos los agentes tienen los
siguientes deberes:

a. Prestar el servicio personalmente, encuadrando su cumplimiento en
principios de eficiencia y eficacia, capacitándose para ello y de
acuerdo a las condiciones y modalidades que resultan del presente
convenio.

b. Observar una actitud ética acorde con su calidad de empleado
universitario y conducirse con respeto y cortesía en sus relaciones con
el público y el resto del personal.

c. Responder por la eficacia y el rendimiento de la gestión del personal
del área a su cargo.

d. Respetar y hacer cumplir, dentro del marco de competencia de su
función, el sistema jurídico vigente.

e. Obedecer toda orden emanada del superior jerárquico competente que
reúna las formalidades del caso y tenga por objeto la realización de
actos de servicio compatibles con la función del agente.

f. Observar el deber de fidelidad que se derive de la índole de las
tareas que le fueran asignadas y guardar la discreción correspondiente,
con respecto a todos los hechos e informaciones de los cuales tenga
conocimiento en el ejercicio o con motivo del ejercicio de sus
funciones, sin perjuicio de lo que establezcan las disposiciones
vigentes en materia de secreto o reserva administrativa.

g. Llevar a conocimiento de sus superiores todo acto, omisión o
procedimiento que causare o pudiere causar perjuicio a la institución
universitaria, configurar delito, o resultar una aplicación ineficiente
de los recursos públicos. Cuando el acto, omisión o procedimiento
involucrare a su superiores inmediatos, podrá hacerlo conocer
directamente a las autoridades de la Institución Universitaria o
denunciarlo al órgano judicial competente.

h. Concurrir a la citación por la instrucción de un sumario, teniendo
obligación de prestar declaración sólo cuando se lo requiera en calidad
de testigo.

i. Someterse a examen psicofísico, en la forma que determine la
reglamentación.

j. Permanecer en el cargo en caso de renuncia, por el término de treinta
días corridos, si antes no fuera reemplazado o aceptada su dimisión o
autorizado a cesar en sus funciones.

k. Excusarse de intervenir en toda actuación que pueda originar
interpretaciones de parcialidad.

I. Velar por el cuidado y la conservación de los bienes que integran el
patrimonio de la Institución Universitaria y los de terceros que
específicamente se pongan bajo su custodia.

m. Seguir la vía jerárquica correspondiente en las peticiones y
tramitaciones realizadas, salvo lo preceptuado en el inciso g.

n) Encuadrarse en las disposiciones legales y reglamentarias sobre
incompatibilidad y acumulación de cargos.

o) Declarar bajo juramento su situación patrimonial y modificaciones
ulteriores en los casos que así se disponga.

Art. 13º: Sin perjuicio de las prohibiciones que en función de las
particularidades de la actividad desempeñada pudieran agregarse en los
respectivos convenios particulares, todos los agentes quedan sujetos a
las siguientes prohibiciones:

a) Patrocinar trámites o gestiones administrativas referentes a asuntos
de terceros que se vinculen con sus funciones.

b) Dirigir, administrar, asesorar, patrocinar, representar o prestar
servicios remunerados o no, a personas de existencia visible o jurídica
que gestionen o exploten concesiones o privilegios de la Institución
Universitaria a la que pertenezca o que fueran sus proveedores o
contratistas.

c) Recibir directa o indirectamente beneficios originados en contratos,
concesiones o franquicias que celebre u otorgue la institución
universitaria.

d) Valerse directa o indirectamente de facultades o prerrogativas
inherentes a sus funciones para fines ajenos a dicha función o para
realizar proselitismo o acción política.

e) Aceptar dádivas, obsequios u otros beneficios u obtener ventajas de
cualquier índole, con motivo u ocasión del desempeño de sus funciones.

f) Representar y/o patrocinar a litigantes o intervenir en gestiones
judiciales o extrajudiciales contra la Institución Universitaria a la
que pertenezca.

g) Desarrollar toda acción u omisión que suponga discriminación por
razón de raza, religión, nacionalidad, opinión, sexo o cualquier otra
condición o circunstancia personal o social.

h) Hacer uso indebido o con fines particulares del patrimonio
universitario.

Art. 14º: Del empleador: Sin menoscabo de las obligaciones emergentes de
otras cláusulas del presente convenio y de los convenios particulares,
son obligaciones del empleador:

a) Observar las normas legales sobre higiene y seguridad en el trabajo,
así como las disposiciones sobre pausas y limitaciones a la duración del
trabajo establecidas en la legislación vigente y el presente convenio.

b) Garantizar al trabajador ocupación efectiva, de acuerdo con su
calificación laboral, salvo por razones fundadas que impidan cumplir
esta obligación.

c) Cumplir con las obligaciones que resulten de las leyes, este convenio
colectivo y de los sistemas de seguridad social, de modo de posibilitar
al trabajador el goce íntegro y oportuno de los beneficios que tales
disposiciones le acuerdan.

d) Depositar en tiempo y forma los fondos correspondientes a la
seguridad social y aportes sindicales a su cargo así como aquellos en
los que actúe como agente de retención.

e) Entregar al trabajador, al extinguirse la relación laboral o durante
ésta cuando medien causas razonables, un certificado de trabajo
conteniendo las indicaciones sobre el tiempo de la prestación de
servicios, naturaleza de éstos, calificación laboral alcanzada, nivel de
capacitación acreditada, constancia de los sueldos recibidos y de los
aportes y contribuciones efectuados con destino a los organismos de
seguridad social.

f) Reintegrar al trabajador los gastos incurridos por éste para el
cumplimiento adecuado del trabajo, que hayan sido previamente
autorizados por autoridades competentes.

g) Garantizar la dignidad del trabajador en el ámbito laboral así como
la no arbitrariedad en la aplicación de sistemas de controles personales
destinados a la protección de los bienes de la Institución
Universitaria.

h) Abstenerse de disponer modificaciones en las condiciones o
modalidades de la relación laboral, con el objeto de encubrir la
aplicación de sanciones.

i) Garantizar la formación en el trabajo, en condiciones igualitarias de
acceso y trato.

j) Informar mensualmente a los organismos sindicales signatarios, en
forma fehaciente, las alteraciones en la situación de revista que se
operen respecto de su padrón de afiliados, y que incidan en sus derechos
y obligaciones sindicales.

Art. 15º: Principios generales: Las partes acuerdan como criterio y
principio básico de interpretación, al que deberán ajustarse las
relaciones laborales del personal comprendido dentro del presente
convenio colectivo de trabajo, el de alcanzar resultados en un ámbito
laboral que permita la evolución y el desarrollo personal del
trabajador, bajo justas y adecuadas condiciones de trabajo y digna
remuneración. En todos los casos se preservará la dignidad del
trabajador, por lo que las funciones y tareas que se mencionan en el
presente convenio colectivo de trabajo deberán interpretase en todos los
casos según los principios de solidaridad y colaboración, que aseguren
continuidad, seguridad, calidad y eficiencia en el servicio público que
prestan las Instituciones Universitarias nacionales.

La aplicación de estos principios no podrá efectuarse de manera que
comporte una disminución salarial o un ejercicio irrazonable de esta
facultad, o cause un perjuicio material o moral al trabajador, de
conformidad con lo establecido en la legislación vigente, ni responda a
formas ocultas o indirectas de sanción.

El empleador deberá capacitar al personal para que haga uso de sus
capacidades para desarrollar diversas tareas, oficios o roles, que se
requieran para poder cumplir con la misión asignada, ya sea en forma
accesoria, complementaria o afín.

Art. 16º: Se crea la Comisión de Categorías Profesionales, Plantas
Normativas y Estructura Salarial integrada por tres (3) representantes
de cada una de las partes signatarias del presente Convenio y que tiene
como misiones y funciones:

a. Analizar y adecuar las situaciones de revista y las funciones
efectivamente desempeñadas, en el marco del régimen escalafonario
vigente.

b. Proponer un Sistema Universitario Nacional de Categorías (SUNC) que
identifique el tipo de labor de cada categoría profesional y una
estructura salarial compatible con los acuerdos alcanzados.

Art. 17º: En el marco de los principios generales precedentemente
expuestos, todo trabajador no docente podrá desempeñar cualquier tarea
en igual o mayor categoría que la que detente, preservando la jerarquía
obtenida. En el supuesto de que por razones debidamente fundadas fuere
necesario para la mejor marcha de la Institución, el cambio de tareas
sólo podrá ser ordenado de acreditarse un proceso de capacitación
direccionada o práctica laboral atinente a la nueva tarea a desempeñar.
En los casos que la aplicación de este principio de por resultado el
ejercicio de una función que cuente con una remuneración mayor de la que
tenía en el puesto anterior, recibirá un suplemento salarial acorde a
esta diferencia y cambio de responsabilidad por el lapso que desempeñe
tal función y sin que ello siente precedente. Si el plazo de permanencia
en la nueva función excediera el año y el cargo estuviese vacante,
deberán ponerse en marcha los mecanismos previstos en el Capítulo de
concursos.

Art. 18º: Los agentes que se vean afectados por medidas de
reestructuración que supriman dependencias, o eliminen o cambien las
funciones asignadas a alguna de ellas, provocando la eliminación de
cargos, serán reubicados en otra función acorde con los conocimientos
adquiridos y la jerarquía obtenida, en las condiciones reglamentarias
que se establezcan al tiempo de resolverse la reestructuración. Para
ello se tomará en cuenta la ocupación de cargos vacantes así como
acciones de reconversión laboral que favorezcan su reinserción.

Art. 19º: Las dependencias suprimidas y los cargos o funciones
eliminados no podrán ser creados nuevamente, ni con la misma
denominación ni con otra distinta por un plazo de dos años a partir de
la fecha de su supresión.

En ningún caso los cargos o funciones eliminados podrán ser cumplidos
por personal contratado, ni por otro de planta subrogado.

Art. 20º: El cambio de tareas se sujetará a la reglamentación que se
acuerde en cada paritaria particular, y en función de las necesidades de
cada Institución Universitaria, respetando los principios
precedentemente enunciados. En atención a la preservación del empleo se
establecerá una red de intercambio de requerimientos laborales, a fin de
facilitar desempeños temporarios o permanentes en distintas
Instituciones Universitarias nacionales. La implementación de este
sistema se acordará en las paritarias particulares, y en cada caso de
traslado se deberá contar con la opinión favorable de ambas
Instituciones Universitarias nacionales y del trabajador.

\subsection{TÍTULO 3}\label{tuxedtulo-3}

\hyperdef{}{condiciones-para-el-ingreso-y-egreso}{\subsubsection{CONDICIONES
PARA EL INGRESO Y EGRESO}\label{condiciones-para-el-ingreso-y-egreso}}

Art. 21º: Para ingresar como trabajador de una Institución Universitaria
nacional se requieren las condiciones de conducta e idoneidad para el
cargo de que se trate, lo que se acreditará a través de los mecanismos
que se establezcan, cumplir satisfactoriamente con el examen de aptitud
psicofísica correspondiente y no estar incurso en alguna de las
circunstancias que se detallan a continuación:

a) Haber sido condenado por delito doloso, hasta el cumplimiento de la
pena privativa de la libertad, o el término previsto para la
prescripción de la pena.

b) Haber sido condenado por delito en perjuicio de cualquier Institución
Universitaria nacional o de la Administración pública nacional,
provincial o municipal.

c) Estar inhabilitado para el ejercicio de cargos público.

d) Haber sido sancionado con exoneración o cesantía en cualquier
Institución Universitaria nacional o en la Administración pública
nacional, provincial o municipal.

e) Haber incurrido en actos de fuerza contra el orden institucional y el
sistema democrático, conforme lo previsto en el artículo 36 de la
Constitución Nacional y el Título X del Código Penal, aún cuando se
hubieren beneficiado por el indulto o la condonación de la pena.

Art. 22º: La relación de empleo del agente con la Institución
Universitaria concluye por las siguientes causas:

a) Renuncia aceptada o vencimiento del plazo para la aceptación expresa
por parte de la autoridad competente según la norma aplicable.

b) Jubilación ordinaria o por invalidez.

c) Aplicación de sanciones de cesantía o exoneración.

d) Retiro voluntario, en los casos que excepcionalmente se establezca.

e) Fallecimiento.

f) Por vencimiento del plazo previsto en el inciso b) del artículo 109º,
o ejercicio de la opción otorgada por su inciso c).

g) Conclusión o rescisión del contrato en el caso del personal no
permanente.

Art. 23º: La jubilación o retiro, la intimación a jubilarse y la
renuncia se regirán por la normativa vigente en la materia.

\subsection{TÍTULO 4}\label{tuxedtulo-4}

\hyperdef{}{regimen-de-concursos}{\subsubsection{REGIMEN DE
CONCURSOS}\label{regimen-de-concursos}}

Art. 24º: El presente título regula las pautas generales de los
procedimientos de selección de personal no docente para la cobertura de
puestos de trabajo, tanto para el ingreso como para la promoción. La
reglamentación respectiva se acordará en el ámbito de las paritarias
particulares.

Art. 25º: Los llamados a concurso serán dispuestos por resolución de la
autoridad facultada para efectuar designaciones, estableciéndose en el
mismo acto quienes se desempeñarán como jurados.

Art. 26º. Clases de concursos: Los concursos podrán ser cerrados o
abiertos. Los concursos cerrados serán a su vez internos o generales,
según participen el personal de planta permanente de la dependencia
solamente o el de toda la institución universitaria, cualquiera fuera la
dependencia. Será concurso abierto aquel en el que puede participar
cualquier persona que reúna los requisitos para el puesto de trabajo a
cubrir.

Art. 27º: El llamado a concurso se publicará en todas las dependencias
de la Institución Universitaria con una antelación mínima de quince (15)
días hábiles a la fecha de apertura de la inscripción; en el caso de que
sea abierto o general se deberá contar con la máxima difusión posible,
mediante la utilización de medios masivos de comunicación apropiado al
lugar de asiento de la Institución Universitaria, lo que incluirá al
menos un diario local. Tratándose de concursos internos, deberá
utilizarse avisos, murales, carteles y los transparentes habilitados a
tal electo. La inscripción se recibirá durante cinco (5) días hábiles.

Art. 28º: En los llamados a concurso deberá especificarse como mínimo lo
siguiente:

a) Clase de concurso, dependencia y jerarquía del cargo a cubrir.

b) Cantidad de cargos a cubrir, horario previsto, remuneración, y
bonificaciones especiales que correspondieren al cargo, si existieran.

c) Requisitos, condiciones generales y particulares exigibles para
cubrir el cargo, con indicación del lugar donde se podrá obtener mayor
información.

d) Lugar, fecha de apertura y cierre de inscripción y entrega de los
antecedentes.

e) Lugar, fecha y hora en que se llevará a cabo la prueba de oposición,
la que deberá tomarse al menos tres (3) días después del cierre de la
inscripción.

f) Temario general.

g) Nombre de los integrantes del jurado.

Cada institución universitaria determinará cuáles aspectos deberán ser
incluidos en la publicidad.

Art. 29º: La asociación gremial del personal no docente de la
Institución Universitaria o cualquier interesado con interés legítimo,
podrá formular observaciones e impugnar el llamado a concurso, dentro
del plazo fijado para la inscripción (art. 27º), cuando éste no se
ajuste a las normas del Convenio Colectivo y a las del presente régimen,
debiendo observar a tal fin las normativas que rigen el procedimiento
administrativo y que resultaren aplicables conforme a la naturaleza de
la cuestión.

Art. 30º: Jurados: Los jurados se constituirán como máximo por cinco
miembros y por no menos de tres. Su integración será resuelta en
paritarias particulares.

Art. 31º: Veeduría: En la oportunidad prevista en el art. 25º serán
convocadas las organizaciones gremiales a participar en carácter de
veedores, designando a un representante. La apertura del concurso deberá
ser notificada en forma fehaciente y tendrán derecho a participar de
todos los actos concursales. Siendo esta participación voluntaria, su
falta no inhabilitará la prosecución del proceso. Podrán observar
solamente cuestiones atinentes a la regularidad del procedimiento.

Art. 32º: Operado el cierre de la inscripción, y una vez verificado el
cumplimiento por parte de los presentados de los requisitos exigidos, se
hará pública la nómina de aspirantes en toda la institución
universitaria a través de las carteleras, y especialmente en la
dependencia a la que corresponda el puesto a concursar, durante cinco
(5) días hábiles. Durante ese lapso, se correrá vista de la
documentación presentada por los otros aspirantes, pudiendo observarla o
impugnarla, durante el mismo lapso. En ese período podrán recusar a los
integrantes del Jurado, y éstos excusarse.

Art. 33º: Sólo se admitirán recusaciones o excusaciones con expresión de
alguna de las causas enumeradas a continuación:

a) El parentesco por consanguinidad dentro del cuarto grado y segundo de
afinidad o la condición de cónyuge entre un Jurado y algún aspirante.

b) Tener el Jurado, su cónyuge o sus consanguíneos o afines, dentro de
los grados establecidos en el inciso anterior, sociedad o comunidad de
intereses con algunos de los aspirantes.

c) Tener el Jurado causa judicial pendiente con el aspirante.

d) Ser el Jurado o aspirante, recíprocamente, acreedor, deudor o fiador.

e) Ser o haber sido el jurado autor de denuncias o querellas contra el
aspirante, o denunciado o querellado por éste ante los Tribunales de
Justicia o autoridades universitarias, con anterioridad a su designación
como Jurado.

f) Haber emitido el Jurado opinión, dictamen o recomendación que pueda
ser considerado como prejuzgamiento acerca del resultado del concurso
que se tramita.

g) Tener el Jurado amistad o enemistad con alguno de los aspirantes que
se manifieste por hechos conocidos en el momento de su designación.

h) Trasgresión por parte del Jurado a la ética universitaria o
profesional, de acuerdo con lo establecido en el artículo siguiente.

Se aplicará subsidiariamente lo dispuesto respecto de recusaciones y
excusaciones en el Código Procesal Civil y Comercial de la Nación. La
resolución que se dicte será irrecurrible.

Art. 34º: Dentro del mismo plazo fijado en el art. 32º, los aspirantes y
los miembros de la comunidad universitaria tendrán derecho a objetar
ante la autoridad que formuló el llamado a los postulantes inscriptos
debido a su carencia de integridad moral, rectitud cívica, ética
universitaria o profesional, o por haber tenido participación directa en
actos o gestiones que afecten el respeto a instituciones, de la
República y a los principios democráticos consagrados por la
Constitución. Estas carencias no podrán ser reemplazadas por méritos
inherentes a las funciones. Serán también causas de objeción, aquellas
que se encuentren comprendidas en las causales de inhabilitación para el
desempeño de cargos públicos.

Art. 35º: Cualquier objeción formulada a los aspirantes o al jurado
deberá estar explícitamente fundada y acompañada de las pruebas que
pretendiera hacerse valer, y especialmente en el caso del artículo
anterior con el fin de eliminar toda discriminación ideológica o
política, de creencia, sociales y culturales.

Art. 36º: Dentro de los dos (2) días hábiles de presentada una
observación, recusación, o impugnación la autoridad competente correrá
traslado al involucrado, quien tendrá un plazo de cinco (5) días hábiles
para formular el pertinente descargo, y ofrecer la prueba de que intente
valerse, lo que deberá hacerse por escrito.

Art. 37º: Efectuado su respectivo descargo o vencido el plazo para
hacerlo, y producida la prueba que hubiere resultado admitida, la
autoridad que efectuara el llamado a concurso tendrá un plazo de cinco
(5) días hábiles para dictar la Resolución pertinente, la quo será
notificada dentro de los dos días hábiles a las partes. Esta resolución
será irrecurrible. En igual plazo admitirá las excusaciones.

Art. 38º: Si la causal de impugnación, recusación, excusación u
observación fuere sobreviniente o conocida con posterioridad, podrá
hacerse valer antes de que el jurado se expida.

Evaluación

Art. 39º: Principios: Los sistemas de evaluación se sujetarán a los
siguientes principios:

a) Objetividad y confiabilidad,

b) Validez de los instrumentos a utilizar,

c) Distribución razonable de las calificaciones en diferentes posiciones
que permitan distinguir adecuadamente los desempeños inferiores, medios
y superiores.

Art. 40º: Del puntaje máximo posible, los antecedentes no podrán
importar más del 50\%, quedando el resto para la prueba de oposición,
proporciones que deberán ser establecidas en la resolución que llame a
concurso, tomando en cuenta las características del cargo a cubrir.

En la evaluación de los antecedentes, la antigüedad podrá valorarse con
hasta un 20\% del total del porcentaje asignado a los antecedentes. Si
hubiera habido evaluaciones de desempeño, la merituación de la
antigüedad estará en función de su resultado, asignándose a la par un
año con un punto solamente cuando la evaluación de ese año haya sido
igual o superior a 5 puntos sobre 10 posibles.

El porcentaje restante de los antecedentes, deberá otorgar una mayor
valoración para los títulos de grado, como así también para el título de
la tecnicatura en gestión universitaria o los cursos de formación
profesional que el aspirante haya presentado, correspondientes a la
función que se evalúa; debiendo aplicarse idéntico criterio al ser
considerados los antecedentes de la función específica; siendo, por
último, considerados los afines. En todos los casos, las paritarias
particulares reglamentarán de acuerdo a los criterios locales los
puntajes correspondientes, los que luego serán utilizados en todos los
casos.

Art. 41º: El jurado deberá dejar constancia de lo actuado en un acta,
que incluirá la consideración de las observaciones o impugnaciones a los
antecedentes efectuadas por los otros aspirantes, el dictamen
debidamente fundado, indicando el orden de mérito de quienes se
encuentren en condiciones de ocupar el puesto concursado, y el listado
de los participantes que no reúnan las condiciones mínimas para ello. Se
considerará en esta situación el aspirante que no reúna el 50\% del
total de puntos posibles. El orden de mérito no podrá consignar empate
en una misma posición y grado.

Todas las decisiones del jurado, incluido el orden de mérito, se tomarán
por mayoría simple de los miembros integrantes del Jurado.

El orden de mérito establecido tendrá un plazo de vigencia de un año, a
contar desde la fecha del dictamen del Jurado.

Art. 42º: Recibido el dictamen del Jurado, la autoridad competente
podrá, dentro de los diez (10) días:

a) Aprobar el dictamen

b) Pedir ampliación de los fundamentos del dictamen.

c) Anular el concurso por defecto de forma o de procedimiento, o por
manifiesta arbitrariedad.

En el mismo acto considerará las observaciones a las que se refiere el
artículo anterior.

Art. 43º: Los concursos serán declarados desiertos en caso de no haber
inscriptos o de insuficiencia de méritos de los candidatos presentados,
lo que dará lugar a un nuevo llamado a concurso.

Designaciones.

Art. 44º: Una vez cumplidos los pasos establecidos en los artículos
anteriores, y dentro de los quince (15) días hábiles dé la última
actuación, la autoridad que corresponda procederá a la designación de
los aspirantes que hubieran ganado el concurso.

Art. 45º: El postulante designado deberá tomar posesión del cargo dentro
de los quince (15) días hábiles de la notificación del respectivo acto
resolutorio, salvo causas justificadas que evaluará la autoridad que lo
designó. En este caso deberán tenerse en cuenta las razones expresadas,
el plazo por el cual se postergará la toma de posesión, y si ello no
entorpece el trabajo para el que se lo hubiera convocado. Si se tratase
de un concurso de ingreso a la Institución Universitaria, para tomar
posesión del cargo deberá haber completado el examen de aptitud
psicofísica.

Art. 46º: Vencido aquel término sin haberse efectivizado la toma de
posesión, o no habiéndose aceptado la causal de la demora, la
designación quedará sin efecto, quedando inhabilitado el concursante
para presentarse a un nuevo concurso en la misma Institución
universitaria, por el plazo de un año. Será designado en este caso el
concursante que siga en el orden de méritos.

\subsection{TÍTULO 5}\label{tuxedtulo-5}

\subsubsection{AGRUPAMIENTOS Y
RETRIBUCIONES}\label{agrupamientos-y-retribuciones}

Agrupamientos

Art. 47º: La estructura salarial del presente Convenio Colectivo de
Trabajo está constituido por cuatro (4) agrupamientos, los que estarán
divididos en tramos y un total de siete (7) categorías.

El alcance y contenido de lo precedentemente detallado será de acuerdo a
las siguientes definiciones:

1) Agrupamientos: Es el conjunto de categorías, divididas en tramos,
abarcativos de funciones programadas para el logro de un objetivo común,
dentro del cual se desarrolla una carrera administrativa. Los
agrupamientos son:

a) Administrativo

b) Mantenimiento, producción y servicios generales

c) Técnico-profesional

d) Asistencial

2) Tramos: Son las partes en que está dividido cada agrupamiento, de
acuerdo a la jerarquía de las funciones cumplidas. Los tramos serán
mayor, intermedio e inicial, con la especificación de funciones que en
cada agrupamiento se establece, y podrá incluir cada uno las categorías
que se indican a continuación:

a) Tramo Mayor: categorías 1, 2 y 3

b) Tramo Intermedio: categorías 4 y 5

c) Tramo Inicial: categorías 6 y 7

3) Categorías: Es cada uno de los niveles jerárquicos de cada
agrupamiento. A cada categoría le corresponden funciones especificas.

4) Cargo: Es la posición concreta del agente en la planta no docente de
la Institución Universitaria, que importa un conjunto de funciones,
atribuciones y responsabilidades, conforme a lo previsto en las
respectivas estructuras orgánico funcionales y que corresponde a cada
trabajador según su categoría de revista.

\paragraph{Agrupamientos}\label{agrupamientos}

Art. 48º: Agrupamiento Administrativo: Este agrupamiento incluirá al
personal que desempeñe funciones de dirección, coordinación,
planeamiento, organización, fiscalización, supervisión, asesoramiento y
ejecución de tareas administrativas, con exclusión de las propias de
otros agrupamientos.

Comprenderá tres (3) tramos, de acuerdo con la naturaleza de las
funciones que para cada uno de ellos se establece, con un total de siete
(7) categorías:

a) Tramo Mayor: incluirá a los trabajadores que cumplan tareas de
dirección, coordinación, planeamiento, organización o asesoramiento,
destinadas a contribuir en la formulación de políticas y planes de
conducción y en la preparación y control de programas y proyectos
destinados a concretar aquéllas. Estará constituido por las categorías
1, 2 y 3.

b) Tramo Intermedio: incluirá a los trabajadores que desarrollen
funciones de colaboración y apoyo al personal del tramo mayor, así como
la supervisión directa de tareas propias del personal del tramo inicial.
Estará constituido por las categorías 4 y 5.

c) Tramo Inicial: incluirá a los trabajadores que desarrollen tareas de
carácter operativo, auxiliar o elemental, estará constituido por las
categorías 6 y 7.

Art. 49º. Agrupamiento Mantenimiento, Producción y Servicios Generales:
Este agrupamiento incluirá al personal que tenga a su cargo tareas de
producción, mantenimiento o conservación de bienes, vigilancia, limpieza
de locales y edificios públicos, manejo de equipos y vehículos
destinados al servicio y las que impliquen atención a otros agentes y al
público.

Comprenderá tres (3) tramos de acuerdo con la naturaleza de las
funciones que para cada uno de ellos se establece, con un total de seis
(6) categorías:

a) Tramo Mayor: incluirá a los trabajadores que administren, programen y
controlen actividades sectoriales. Se integrará con las categorías 2 y
3.

b) Tramo Intermedio: Incluirá a los trabajadores que ejerzan funciones
de colaboración y apoyo al personal del tramo mayor y de supervisión y
control de las tareas encomendadas al personal del tramo inicial; o
realicen funciones específicas o especializadas. Se integrará con las
categorías 4 y 5.

c) Tramo Inicial: Incluirá a los trabajadores que desarrollen tareas de
carácter operativo, auxiliar o elemental. Se integrará con las
categorías 6 y 7.

Art. 50º. Agrupamiento Técnico - Profesional: Este agrupamiento incluirá
a los trabajadores que desempeñen funciones de las siguientes
características:

A.- Profesionales, que abarcará aquellas para las cuales sea requisito
poseer título universitario, y que consistan específicamente en el
ejercicio de sus incumbencias profesionales.

Comprenderá dos (2) tramos, de acuerdo a la naturaleza de las funciones
que para cada uno de ellos se establezcan con un total de cinco (5)
categorías:

a) Tramo Mayor: Incluirá a los trabajadores que realicen funciones de
programación profesional, jefatura, administración, control del área de
su competencia, ejecución de tareas de nivel superior. Estará
constituido por las categorías 1, 2 y 3.

b) Tramo Intermedio: Incluirá a los trabajadores que desempeñen
funciones de colaboración y apoyo profesional especializadas, así como
la supervisión directa de tareas específicas del tramo inicial. Estará
constituido por las categorías 4 y 5.

B.- Técnicas, que abarcará aquellas para las cuales sea requisito poseer
título habilitante. En casos en que en la especialidad requerida no se
otorguen títulos específicos, o no hubiera en el lugar alguien que lo
posea, este requisito podrá ser reemplazado por la demostración de la
idoneidad adecuada para el desempeño de las funciones técnicas
requeridas.

Comprenderá tres (3) tramos, de acuerdo a la naturaleza de las funciones
que para cada uno de ellos se establezcan con un total de seis (6)
categorías:

a) Tramo Mayor: Incluirá a los trabajadores que realicen funciones de
programación técnica, jefatura, administración, control técnico del área
de su competencia, ejecución de tareas de nivel superior. Estará
constituido por las categorías 2 y 3.

b) Tramo Intermedio: Incluirá a los trabajadores que desempeñen
funciones de colaboración y apoyo técnico especializadas, así como la
supervisión directas de tareas específicas del tramo inicial. Estará
constituido por las categorías 4 y 5.

c) Tramo Inicial: Incluirá a los trabajadores que ejecuten tareas de
carácter técnico operativo, conforme a la capacitación y experiencia
adquiridas en su especialidad. Estará constituido por las categorías 6 y
7.

Art. 51º. Agrupamiento Asistencial: Este agrupamiento incluirá a los
trabajadores que presten servicio en unidades hospitalarias,
académicas-asistenciales, y laboratorios que contribuyan al tratamiento
de la salud.

Estará subdividido en cuatro (4) subgrupos de acuerdo con las funciones
que desempeñen y en cada uno de ellos se establecen, con un total de
tres (3) tramos y hasta siete (7) categorías:

Subgrupo ``A'': incluirá a los trabajadores que posean título
universitario y desempeñen funciones propias de su incumbencia
profesional, en tareas de dirección, coordinación, planeamiento y
organización hospitalaria, académica, sanitaria o asistenciales y de
atención directa al paciente. Abarcará a los médicos, odontólogos,
bioquímicos, farmacéuticos, profesionales equivalentes, kinesiólogos,
técnicos de laboratorios, equivalentes y funciones auxiliares.
Comprenderá tres (3) tramos, de acuerdo a la naturaleza de las funciones
que para cada uno de ellos se establezcan con un total de cinco (5)
categorías:

a) Tramo Mayor: Incluirá a los trabajadores que realicen funciones de
programación técnicas y/o profesionales, jefatura, administración,
control técnico del área de su competencia, y ejecución de tareas de
nivel superior. Estará constituido por las categorías 2 y 3.

b) Tramo Intermedio: Incluirá a los trabajadores que desempeñen
funciones de colaboración y apoyo técnico y/o profesional
especializadas, así como la supervisión directa de tareas específicas
del tramo básico. Estará constituido por las categorías 4 y 5.

c) Tramo Inicial: Incluirá a los trabajadores que ejecuten tareas de
carácter técnico y/o profesional, conforme a la capacitación y
experiencia adquiridas en su especialidad. Estará constituido por la
categoría 6.

Subgrupo ``B'': Incluirá a los trabajadores que desempeñen funciones de
enfermería en tareas de dirección, organización, jefatura, supervisión,
ejecución o auxiliar. Comprenderá tres (3) tramos, de acuerdo a la
naturaleza de las funciones que para cada uno de ellos se establezcan
con un total de seis (6) categorías:

a) Tramo Mayor: Incluirá a los trabajadores que realicen funciones de
dirección, programación, jefatura, administración, control del área de
su competencia. Ejecución de tareas de nivel superior. Estará
constituido por las categorías 2 y 3.

b) Tramo Intermedio: Incluirá a los trabajadores que desempeñen
funciones de colaboración y tareas especializadas, así como la
supervisión directa de tareas específicas del tramo básico. Estará
constituido por las categorías 4 y 5.

c) Tramo Inicial: Incluirá a los trabajadores que ejecuten tareas
conforme a la capacitación y experiencia adquiridas en su especialidad.
Estará constituido por las categorías 6 y 7.

Subgrupo ``C'': Incluirá a los trabajadores que desempeñen funciones de
dirección, coordinación, planeamiento, organización, fiscalización,
supervisión, asesoramiento, ejecución de tareas administrativas, con
exclusión de las propias de otros subgrupos.

Comprenderá tres (3) tramos, de acuerdo con la naturaleza de las
funciones que para cada uno de ellos se establece, con un total de siete
(7) categorías:

a) Tramo Mayor: incluirá a los trabajadores que cumplan tareas de
Dirección, Coordinación, Planeamiento, Organización, Control o
Asesoramiento, destinadas a contribuir en la formulación de políticas y
planes de conducción y en la preparación, ejecución y control de
programas y proyectos destinados a concretar aquellas. Estará
constituido por las categorías 1, 2 y 3.

b) Tramo Intermedio: incluirá a los trabajadores que desarrollen
funciones de colaboración y apoyo al personal de Dirección, así como la
supervisión directa de tareas propias del personal de ejecución. Estará
constituido por las categorías 4 y 5.

c) Tramo Inicial: incluirá a los trabajadores que desarrollen tareas de
carácter operativo, auxiliar o elemental, estará constituido por las
categorías 6 y 7.

Subgrupo ``D'': Incluirá a los trabajadores que tengan a su cargo tareas
de producción, mantenimiento o conservación de bienes, vigilancia,
limpieza de locales y edificios públicos, manejo de equipos y vehículos
destinados al servicio y las que impliquen atención a otros agentes y al
público.

Comprenderá tres (3) tramos de acuerdo con la naturaleza de las
funciones que para cada uno de ellos se establece, con un total de cinco
(5) categorías:

a) Tramo Mayor: incluirá a los trabajadores que cumplan tareas de
Dirección, Coordinación, Planeamiento, Organización, Control o
Asesoramiento, destinadas a contribuir en la formulación de políticas y
planes de conducción y en la preparación, ejecución y control de
programas y proyectos destinados a concretar aquéllas. Estará
constituido por la categoría 3.

b) Tramo Intermedio: Incluirá a los trabajadores que ejerzan funciones
de supervisión y control de las tareas encomendadas al personal del
tramo básico y la realización de funciones específicas o especializadas.
Se integrará con las categorías 4 y 5.

c) Tramo Inicial: Incluirá a los trabajadores que desarrollen tareas de
carácter operativo, en relación de dependencia con las jerarquías del
tramo medio se integrará con las categorías 6 y 7.

\hyperdef{}{retribuciones}{\paragraph{Retribuciones}\label{retribuciones}}

Art. 52º: La retribución del trabajador no docente se compone del sueldo
básico correspondiente a su categoría; los adicionales particulares y
los suplementos que correspondan a su situación de revista y condiciones
generales.

Art. 53º: El Sueldo Básico que hace a la asignación de la Categoría
consistirá en el importe resultante de la aplicación de los índices
expresados, teniendo en cuenta los coeficientes que a continuación se
detallan, y cuyo monto testigo es el coeficiente 1.00 = a la categoría
7.

\begin{longtable}[c]{@{}lll@{}}
\toprule
\begin{minipage}[t]{0.18\columnwidth}\raggedright\strut
CATEGORIAS
\strut\end{minipage} &
\begin{minipage}[t]{0.16\columnwidth}\raggedright\strut
BASICOS
\strut\end{minipage} &
\begin{minipage}[t]{0.18\columnwidth}\raggedright\strut
TRAMOS
\strut\end{minipage}\tabularnewline
\begin{minipage}[t]{0.18\columnwidth}\raggedright\strut
7 6 5 4 3 2 1
\strut\end{minipage} &
\begin{minipage}[t]{0.16\columnwidth}\raggedright\strut
1.00 1.20 1.44 1.73 2.08 2.50 3.00
\strut\end{minipage} &
\begin{minipage}[t]{0.18\columnwidth}\raggedright\strut
INICIAL

INTERMEDIO

MAYOR
\strut\end{minipage}\tabularnewline
\bottomrule
\end{longtable}

\paragraph{Adicionales}\label{adicionales}

Art. 54º: Establécense los siguientes adicionales:

a) Por antigüedad.

b) Por título.

c) Por permanencia en la categoría.

d) Por tarea asistencial.

Art. 55º. Adicional por antigüedad: A partir del 1º de enero de cada
año, el trabajador comprendido en este Convenio percibirá en concepto de
``adicional por antigüedad'' la suma equivalente al UNO POR CIENTO (1\%)
de la asignación de la categoría de revista por cada año de servicio o
fracción mayor de SEIS (6) meses que registre al 31 de diciembre
inmediato anterior.

Art. 56º: La antigüedad de cada trabajador no docente se determinará
sobre la base de los servicios no simultáneos, prestados en forma
ininterrumpida o alternada en organismos nacionales, provinciales o
municipales, inclusive los siguientes:

1) Los prestados en calidad de contratado, siempre que se cumplan los
requisitos que a continuación se especifican:

a) Que sean servicios prestados en relación de dependencia.

b) Que estuvieren sujetos a un determinado horario, susceptible de un
adecuado contralor.

2) Los prestados con carácter ad-honorem, siempre que los mismos sean
fehacientemente comprobados.

3) Los prestados en entidades privadas que hubieren sido incorporadas a
la Nación, Provincia o Municipio, sólo en el caso en que a la fecha de
producirse esa incorporación el agente estuviese prestando efectivamente
servicios en ellas.

4) Los prestados en las Fuerzas Armadas o de Seguridad.

Art. 57º: No se computarán a los efectos del adicional por antigüedad:

1) Los servicios que hubieran originado jubilación, retiro o pensión
cuando el trabajador perciba la correspondiente prestación de pasividad
en forma total o parcial.

2) Los lapsos correspondientes a suspensiones o licencias sin goce de
sueldo, superiores a Treinta (30) días continuos o discontinuos.

Art. 58º: Cuando el trabajador no docente desempeñare más de un empleo
en organismos comprendidos en este Convenio, el cómputo se hará mediante
el siguiente procedimiento:

a) En los distintos empleos se computarán exclusivamente los años de
servicio cumplidos en cada uno de ellos.

b) La antigüedad restante que el trabajador no docente tuviere
acreditada por otras prestaciones, se considerará en el empleo donde
éste tenga mayor antigüedad,

c) Cuando el trabajador cesare en uno de sus empleos, podrá trasladar al
más antiguo de los que mantuviere las prestaciones acreditadas en el que
deja vacante, siempre que no se tratare de servicios simultáneos.

Art. 59º: Cuando el trabajador desempeñare más de un empleo y alguno de
ellos fuere cumplido en organismos excluidos de este Escalafón que
tuvieran implementado un régimen de bonificaciones por antigüedad, sólo
se le reconocerán a los efectos de las presentes normas, los servicios
que no sean ya bonificados en sus otros empleos.

En caso de cesar en éstos, manteniendo únicamente el empleo comprendido
en este Escalafón, se le reconocerá la antigüedad total que acredite.

Los reconocimientos de servicios serán considerados en todos los casos a
partir del día 1º del mes siguiente al de la presentación de sus
respectivas certificaciones.

Art. 60º. Adicional por título: El trabajador no docente de las UUNN,
percibirá el adicional por título de conformidad con las siguientes
normas:

a) Títulos universitarios de carreras de posgrado, el treinta por ciento
(30\%) de la asignación de la categoría de revista.

b) Títulos universitarios de carreras de grado, veinticinco por ciento
(25\%) de la asignación de la categoría de revista.

c) Tecnicatura en Gestión Universitaria, veinte por ciento (20\%) de la
asignación de la categoría de revista.

d) Títulos universitarios de pregrado o de estudios superiores, que
demanden de uno (1) a tres (3) años de estudio de tercer nivel: diez por
ciento (10\%) de la asignación de la categoría de revista.

e) Títulos secundarios en sus distintas especialidades y del polimodal y
los similares expedidos por la Dirección Nacional de Educación del
adulto: diecisiete con cincuenta por ciento (17,50\%) de la asignación
de la categoría 7.

Art. 61º: Los títulos universitarios de estudios superiores que
acrediten una misma incumbencia profesional se bonificarán en igual
forma, aún cuando hubieran sido obtenidos con arreglo a planes de
estudios de distinta duración, teniendo en cuenta la máxima prevista
para la carrera.

Art. 62º: Los títulos cuya posesión se invoque serán reconocidos a
partir del 1º del mes siguiente a la fecha de presentación de las
certificaciones respectivas.

A tales efectos resultarán válidas las certificaciones extendidas por
los correspondientes establecimientos educacionales, por las que se
acredite que el trabajador ha finalizado sus estudios correspondientes
al plan de la carrera y que tiene en trámite el título que así lo
acredita. Sin perjuicio de ello, deberá exigirse al interesado su
presentación en la oportunidad en que aquellossean extendidos.

Art. 63º: No podrá bonificarse más de un título por empleo,
reconociéndose en todos los casos aquel al que le corresponda un
adicional mayor.

Art. 64º. Adicional por permanencia en la categoría: Todos los
trabajadores comprendidos en el presente Convenio percibirán un
adicional a partir de los dos (2) años de permanencia en la categoría,
de hasta un máximo del setenta por ciento (70\%) de la diferencia entre
la asignación básica de la categoría de revista y la de la inmediata
superior, de acuerdo con el siguiente detalle:

\begin{longtable}[c]{@{}ll@{}}
\toprule
\begin{minipage}[t]{0.47\columnwidth}\raggedright\strut
Años de permanencia en la categoría
\strut\end{minipage} &
\begin{minipage}[t]{0.47\columnwidth}\raggedright\strut
\% de la diferencia con la categoría inmediata superior
\strut\end{minipage}\tabularnewline
\begin{minipage}[t]{0.47\columnwidth}\raggedright\strut
2

4

6

8
\strut\end{minipage} &
\begin{minipage}[t]{0.47\columnwidth}\raggedright\strut
10

25

45

70
\strut\end{minipage}\tabularnewline
\bottomrule
\end{longtable}

Este adicional dejará de percibirse cuando el trabajador sea promovido.

Art. 65º: Para el trabajador que reviste en la categoría máxima el
adicional se calculará sobre el treinta y siete por ciento (37\%) de la
asignación de la categoría.

Art. 66º. Adicional por tarea asistencial: El personal perteneciente al
agrupamiento asistencial percibirá un adicional especial equivalente al
doce por ciento (12\%) de la asignación de la categoría de revista.

Art. 67º. Adicional por dedicación exclusiva asistencial: Aquellos
agentes del agrupamiento asistencial que cumplan con los requisitos que
se detallan en el presente artículo, percibirán un adicional especial no
bonificable por dedicación exclusiva, equivalente al veinticinco por
ciento (25\%) de la remuneración básica correspondiente a su categoría
de revista.

Los requisitos que deberán reunir son:

a) Cumplir no menos de cuarenta (40) horas semanales de labor.

b) Haber optado expresamente por el régimen de dedicación exclusiva,
presentando declaración jurada en la que conste la inexistencia de otra
relación de empleo.

c) Acreditar capacitación específica mediante título habilitante o
certificado otorgado por autoridad competente.

Este adicional absorbe al adicional por tarea asistencial.

Suplementos

Art. 68º: Establécense los siguientes suplementos:

a) Por zona desfavorable.

b) Por falla de caja.

c) Por riesgo.

d) Por mayor responsabilidad.

Art. 69º. Suplemento por zona desfavorable: El personal comprendido en
el presente Convenio Colectivo de Trabajo percibirá el suplemento por
zona desfavorable, en los casos, montos y condiciones que establezca la
normativa general pertinente.

Art. 70º. Suplemento por fallas de caja: Este suplemento se liquidará a
los trabajadores no docentes que se desempeñen con carácter regular y
permanente en tareas inherentes al manejo de fondos en efectivo
(pagadores, tesoreros, cajeros o funcionarios similares) o realicen
tareas de recaudación y pago, y consistirá en la suma mensual
equivalente al veinticinco por ciento (25\%) de la asignación de la
categoría 7.

Art. 71º. Suplemento por riesgo: Este suplemento se liquidará a los
trabajadores no docentes que desempeñen funciones cuya naturaleza
implique la realización de acciones o tareas en las que se ponga en
peligro cierto su integridad psico-física.

Las funciones que se consideren incluidas en la percepción de este
suplemento, así como el respectivo importe, deberán establecerse en cada
caso, conforme lo dispuesto por el área de seguridad y riesgos laborales
de la universidad o, supletoriamente, por el ministerio de trabajo de la
nación, y en ningún caso podrá superar el diez por ciento (10\%) del
total de la asignación de la categoría de revista.

Art. 72º. Suplemento por mayor responsabilidad: Este suplemento se
liquidará a los trabajadores no docentes que desarrollen tareas de mayor
responsabilidad que la asignada a la categoría de revista y consistirá
en una suma equivalente a la diferencia de su categoría con la
correspondiente a la jerarquía que le toque desempeñar, en los casos
establecidos en el artículo 17 del presente convenio colectivo de
trabajo.

Sólo se aplicará este suplemento cuando exista la vacante o el titular
del cargo se encuentre con licencia que dé origen a la cobertura de la
mayor responsabilidad. En ningún caso podrá utilizarse este suplemento
para disponer pagos diferenciados donde no existan las circunstancias
antes anotadas.

\paragraph{Tipificador de funciones}\label{tipificador-de-funciones}

Art. 73: La Comisión Negociadora de Nivel General elaborará un
tipificador de funciones que se integrará como anexo del convenio
colectivo.

El re-encasillamiento y re-ubicación escalafonaria del trabajador no
docente, de conformidad con las denominaciones y niveles que integran el
presente convenio colectivo será el que determine la nueva ubicación por
agrupamiento, tramo y categoría que le corresponda a cada uno a partir
de su fecha de vigencia.

Las situaciones que pudieran modificar el encasillamiento original y que
surgieren a partir de recursos presentado ante la comisión de
re-encasillamiento creada a tal efecto, regirán siempre con
retroactividad a la fecha de aquel encasillamiento. Cualquier
interpretación o aclaración complementaria referida al presente convenio
colectivo serán resuelta por la comisión creada al efecto.

\subsection{TÍTULO 6}\label{tuxedtulo-6}

\hyperdef{}{tiempo-de-trabajo}{\subsubsection{TIEMPO DE
TRABAJO}\label{tiempo-de-trabajo}}

\paragraph{Jornada de trabajo}\label{jornada-de-trabajo}

Art. 74º: Se establece la jornada de trabajo convencional de 35 horas
semanales, siete diarias continuas corridas, de lunes a viernes. El
exceso de la jornada habitual de que se trate será considerado como hora
extra convencional con el recargo del 50\% en los días hábiles, y del
100\% en días inhábiles. En ningún caso la jornada podrá extenderse más
de 10 horas. Si hubiera un acuerdo entre el empleador y el trabajador,
se podrán compensar las horas extra trabajadas con una reducción horaria
equivalente, la que deberá operarse durante el mismo mes calendario, o
plazo mayor de hasta un año, cuando por razones de servicio así lo
establezca la Institución Universitaria, de acuerdo al cupo y otras
modalidades. Estas normas no rigen en caso de trabajos cuya organización
horaria haga habitual el cumplimiento de tareas en horarios nocturnos o
días inhábiles, las que serán reglamentadas por la Comisión Negociadora
de Nivel General.

Art. 75º: El personal incluido en el presente convenio podrá solicitar
una reducción horaria de hasta tres horas. La resolución del pedido
quedará a criterio exclusivo de la parte empleadora. En el supuesto de
ser otorgada, la retribución se reducirá en igual proporción, tomando
como base las remuneraciones y adicionales considerados regulares y
permanentes.

Art. 76º: Serán liquidadas con un recargo del 25\% las horas que
correspondan a la extensión horaria que se acuerde entre las partes para
proporcionar capacitación direccionada al trabajador, cuando razones de
servicio aconsejen que se haga fuera de los horarios habituales. Podrán
ser compensadas, en cuyo caso se adoptará el criterio establecido en el
artículo 74º.

Art. 77º: Régimen de descanso: El régimen de descanso se ajustará al
presente convenio colectivo y a las normas legales vigentes.

Art. 78º: El trabajador que cumpla jornada completa continua, tendrá
derecho a una pausa de treinta (30) minutos entre la tercera y cuarta
hora del ingreso. Esta medida se aplicará con modalidad de relevo.

Licencias, justificaciones y franquicias

Licencias. Licencia anual ordinaria

Art. 79º: El trabajador gozará de un período mínimo y continuado de
descanso anual remunerado por los plazos que se establecen:

de 20 días corridos, cuando la antigüedad no exceda los 5 años.

de 25 días corridos, cuando siendo la antigüedad mayor de 5 años no
supere los 10 años.

de 30 días corridos, cuando siendo la antigüedad mayor de 10 años no
supere los 15.

de 35 días corridos, cuando la antigüedad sea mayor de 15 años y no
exceda los 20 años.

de 40 días corridos, cuando la antigüedad sea de 20 años o más.

En los dos últimos casos la licencia anual podrá ser fraccionada en dos
períodos, uno de los cuales deberá ser de al menos treinta días
corridos, siempre que medie acuerdo de partes.

Art. 80º: La licencia comenzará el día lunes o el día siguiente hábil si
aquél fuere feriado; en los casos de los trabajadores que prestan
servicio en días inhábiles, las vacaciones deberán comenzar el día
siguiente a aquel en que el trabajador finalice su descanso semanal o en
el subsiguiente hábil, si aquél fuera feriado.

Art. 81º: Para determinar la extensión de las vacaciones, la antigüedad
en el empleo se computará como aquella que tenga el trabajador,
debidamente acreditada, al 31 de diciembre del año al que correspondan.

Art. 82º: El trabajador tendrá derecho y obligación al goce de la
licencia cada año, habiendo prestado servicio como mínimo durante la
mitad del total de los días hábiles comprendidos en el año calendario. A
este efecto se computarán como hábiles los días feriados trabajados,
como tarea normal. Cuando el trabajador no llegase a totalizar el tiempo
mínimo de trabajo previsto precedentemente para computársele el año
completo, gozará de una licencia de un día por cada veinte de trabajo
efectivamente realizado.

Art. 83º: No se computarán como trabajados a los efectos del artículo
anterior los días de uso de licencias sin goce de haberes.

Art. 84º: La licencia anual ordinaria será otorgada entre el 15 de
diciembre del año al que corresponde y el 28 de febrero del año
siguiente, teniendo en cuenta el período de receso de actividades de la
Institución universitaria. Cada Institución Universitaria podrá disponer
excepciones a esta regla, cuando razones suficientemente fundadas en
necesidades del servicio así lo aconsejen.

Art. 85º: Se dará preferencia en la selección de la fecha de las
vacaciones al trabajador que tenga hijos en edad escolar a su cargo. De
estar empleados en la misma Institución Universitaria ambos cónyuges se
les concederá la licencia anual ordinaria en forma simultánea, salvo
pedido en contrario de los interesados. Se considerará especialmente el
caso en que ambos cónyuges trabajen en distintos ámbitos del sistema
universitario, y el del trabajador que tenga otro empleo, de manera de
facilitar las vacaciones simultáneas en uno y la unificación de los
períodos en el otro.

Art. 86º: La fecha de iniciación de la licencia será comunicada por
escrito, con una anticipación no menor de cuarenta y cinco días
corridos.

Art. 87º: En ningún caso la licencia anual ordinaria podrá ser acumulada
o compensada pecuniariamente, por lo que es responsabilidad de las
partes que sea otorgada y gozada en el período al que corresponda.

Art. 88º. Postergación de la licencia: Cuando el trabajador no haya
podido usufructuar la licencia anual ordinaria en el período en que se
le hubiese otorgado por estar haciendo uso de otra licencia de las aquí
reglamentadas, o bien por estar realizando estudios o investigación
científica, actividades técnicas o culturales autorizadas por la
institución universitaria, gozará la licencia anual ordinaria dentro de
los seis meses de la fecha en que se reintegre al servicio.

Art. 89º. Interrupción de licencia: La licencia anual ordinaria podrá
interrumpirse sólo por cuestiones de salud que exijan una atención
certificada de 5 días o más, por maternidad, fallecimiento de familiar,
atención de hijo menor y por los lapsos correspondientes al presente
régimen de licencias. En estos supuestos se reiniciará el cómputo de la
licencia anual ordinaria una vez finalizadas las causales descriptas
anteriormente. Estos casos no se considerarán como fraccionamiento de la
licencia.

Art. 90º: En caso de cese de la relación de empleo sin que el trabajador
haya gozado de la licencia anual ordinaria, se le liquidará el monto
proporcional correspondiente a la compensación de la licencia no gozada,
de acuerdo a lo establecido en el art. 59º. Igual procedimiento se
llevará a cabo a favor de sus derecho-habientes, los que percibirán el
monto correspondiente.

\paragraph{Licencias por enfermedad}\label{licencias-por-enfermedad}

Art. 91º: Al trabajador que deba atenderse afecciones o lesiones de
corto tratamiento, que inhabiliten para el desempeño del trabajo,
incluidas operaciones quirúrgicas menores, se le concederán hasta
cuarenta y cinco (45) días corridos de licencia por año calendario, en
forma continua o discontinua, con percepción íntegra de haberes. Vencido
este plazo, cualquier otra licencia que sea necesario acordar en el
curso del año por las causales enunciadas, será sin goce de haberes.

Art. 92º: Si por enfermedad el agente debiera retirarse del servicio, se
considerará el día como licencia por enfermedad de corto tratamiento si
hubiera transcurrido menos de media jornada de labor, y permiso personal
o excepcional, cuando hubiere trabajado más de media jornada.

Art. 93º: El trabajador tendrá derecho a una licencia extraordinaria de
hasta un año, con percepción del 100\% de sus haberes por afecciones o
Iesiones de largo tratamiento que lo inhabiliten para el desempeño del
trabajo. Vencido ese plazo, subsistiendo la causal que determinó la
licencia y en forma excepcional, se ampliará este plazo por hasta dos
(2) nuevos períodos de seis meses con percepción del 100\% de haberes,
hasta dos períodos de seis meses más con percepción del 50\% de los
haberes, y otros dos de igual duración sin goce de haberes. Para ello
será necesaria la certificación de la autoridad sanitaria establecida
para estos casos, que comprenda el estado de afección o lesión, la
posibilidad de recuperación y el período estimado de inhabilitación para
el trabajo.

Art. 94º: En caso que el estado de salud del agente lo constituya con
derecho a una jubilación por incapacidad, se iniciarán los trámites de
inmediato, y se le abonará el 95\% del estimado de haber jubilatorio,
hasta que se le otorgue. El importe se liquidará con carácter de
anticipo. El cumplimiento de lo aquí dispuesto quedará supeditado a la
existencia de un convenio con el ANSES que garantice la devolución.

Art. 95º: La enfermedad laboral o el accidente de trabajo quedará
cubierto según lo dispuesto por la Ley de Riesgos del Trabajo, o
normativa que la reemplace, considerándose que el trabajador está en uso
de licencia por los períodos de cobertura. Cuando se tratase de casos de
este tipo que no correspondan a la cobertura de las Aseguradoras de
Riesgos de Trabajo recibirá igual trato que el caso de enfermedades
inculpables o de largo tratamiento, con más la indemnización que le
corresponda.

Art. 96º: Si como resultado de las afecciones mencionadas en los
artículos precedentes se declarase la incapacidad parcial, se requerirá
certificación profesional de autoridad pública que determine el tipo de
funciones que puede desempeñar, como así también el horario a cumplir,
que en ningún caso podrá ser inferior a cuatro horas diarias. Con esta
certificación, la Institución Universitaria adecuará la labor a las
recomendaciones efectuadas, debiendo abonar la retribución total por un
lapso que no podrá extenderse por más de un año. Vencido ese lapso, se
aplicarán las disposiciones relativas a la jubilación por invalidez.

\paragraph{Licencias extraordinarias y
justificaciones}\label{licencias-extraordinarias-y-justificaciones}

Art. 97º: El trabajador gozará de las siguientes licencias especiales:

a) En caso de trabajador varón, por nacimiento u otorgamiento de la
guarda para adopción de hijos, tres días hábiles,

b) Por matrimonio, 10 días hábiles.

c) Por matrimonio de un hijo, 2 días.

d) Por fallecimiento del cónyuge o pariente en primer grado de
consanguinidad, 10 días. Si el deceso que justificase esta licencia
fuera del cónyuge y el trabajador supérstite tuviera hijos menores de
edad, la licencia se extenderá por 15 días más.

e) Por fallecimiento de pariente en segundo grado de consanguinidad, 5
días.

f) Por fallecimiento de pariente político en 1º y 2º grado, un (1) día,
el que coincidirá con el del deceso o el del sepelio.

g) Donación de sangre, un (1) día, el de la extracción.

h) Para rendir examen por enseñanza media, 20 días hábiles por año
calendario con un máximo de 4 días por examen,

i) Para rendir examen por enseñanza superior, 24 días hábiles por año
calendario con un máximo de 5 días por examen.

Las inasistencias producidas por razones de fuerza mayor, fenómenos
meteorológicos y circunstancias de similar naturaleza serán justificadas
por la Institución Universitaria, siempre que se acrediten debidamente o
sean de público y notorio conocimiento.

Las inasistencias en que incurra el trabajador no docente con motivo de
haber sido autorizado a concurrir a conferencias, congresos, simposios
que se celebren en el país con auspicio oficial, sindical o declarados
de interés nacional, serán justificadas con goce de haberes.

Las licencias por actividades deportivas no rentadas serán reconocidas
hasta 15 días por año calendario. Se otorgará en los casos que el agente
asista en Representación oficial, nacional, provincial, municipal,
universitaria o gremial.

Las licencias a que se refiere este artículo serán con goce de haberes,
de acuerdo a la situación de revista y la remuneración que perciba el
trabajador en forma habitual.

Art. 98º: En todos los casos mencionados en el artículo anterior, deberá
acreditarse la circunstancia que justificó la licencia dentro de las 72
horas de producido el reintegro del trabajador; en los referidos en los
inc. b), c), h) e i), deberá además solicitarla con 20 días de
anticipación.

Art. 99º: A los efectos del otorgamiento de las licencias a que aluden
los incisos h) e i) del art. 74º, los exámenes corresponderán a planes
de enseñanza oficial.

Art. 100º. Razones particulares: El trabajador tendrá derecho a hacer
uso de licencia sin goce de haberes en forma continua o fraccionada en
no más de dos períodos, hasta completar 12 meses, dentro de cada
decenio, siempre que el trabajador cuente con una antigüedad mínima de
10 años en la Institución Universitaria y será acordada siempre que no
se opongan razones de servicio. Tendrá igual derecho el trabajador cuyo
cónyuge haya sido designado en una función oficial en el extranjero, o
en la Argentina en lugar distante a más de 100 km. del lugar donde
presta servicios a la Institución Universitaria, y siempre que dicha
función oficial comprenda un período superior a los 90 días.

Art. 101º: Se podrán otorgar hasta seis (6) permisos particulares por
año, con goce de haberes, de una jornada cada uno, para atender trámites
o compromisos personales que no puedan ser cumplidos fuera del horario
de trabajo. En ningún caso podrán acumularse más de dos (2) días en el
mes. Para la utilización de estos permisos el trabajador deberá dar
aviso con 24 horas de antelación, quedando sujeta su autorización a las
necesidades del servicio.

Art. 102º. Permisos excepcionales: Se podrán justificar hasta cinco (5)
permisos excepcionales por año, con goce de haberes, otorgados por el
responsable directo del área donde preste servicio el trabajador,
después de haberse cumplido como mínimo las dos primeras horas de la
jornada de labor correspondiente, y siempre que obedecieran a razones
atendibles y el servicio lo permita.

Art. 103º: Las Paritarias particulares podrán acordar otras licencias
comunes a todo el personal, en épocas del año en que la actividad
académica de cada Institución Universitaria lo permita.

Art. 104º. Atención de familiar enfermo: Los trabajadores incluidos en
el presente convenio están obligados ante la Institución Universitaria a
presentar una declaración jurada, consignando todos los datos de quienes
integran su grupo familiar y de cómo ellos dependen de su atención y
cuidado. El trabajador dispondrá de hasta 20 días corridos, en un solo
período o fraccionado, en el año, con goce de haberes para atender a
alguno de esos familiares que sufra enfermedad o accidente que requiera
la atención personal del trabajador, plazo que podrá extenderse hasta en
100 días adicionales, extensión que será sin goce de haberes. Para la
justificación de estos supuestos deberá presentar la certificación
profesional con identidad del paciente y la referencia explícita a que
requiere atención personalizada, todo lo que será certificado por el
servicio médico de la Institución Universitaria.

Art. 105º: El trabajador que fuera designado o electo para desempeñar
cargos de mayor jerarquía en el orden nacional, provincial o municipal,
queda obligado a solicitar licencia sin percepción de haberes; que se
acordará por el término en que se ejerzan esas funciones.

Tendrán derecho a la reserva de su empleo por parte del empleador, y a
su reincorporación hasta 30 días después de concluido el ejercicio de
aquellas funciones. El período durante el cual haya desempeñado las
funciones aludidas será considerado período de trabajo a los efectos del
cómputo de su antigüedad.

El trabajador electo para desempeñar funciones en la conducción de la
Federación Argentina del Trabajador de Universidades Nacionales (FATUN)
previstas en su estatuto, tendrá derecho a licencia paga. Tendrán este
derecho hasta doce agentes, y no más dé uno por institución
universitaria, por el período correspondiente al desempeño de esas
funciones, conservando el puesto de trabajo hasta treinta (30) días
después de finalizado el mandato para el cual fuera electo, período
dentro del cual deberá reintegrarse.

Art. 106º. Maternidad: La trabajadora deberá comunicar el embarazo al
empleador con presentación del certificado médico en el que conste la
fecha presumible del parto. Queda prohibido el trabajo de personal
femenino dentro de los 45 días anteriores y los 45 días posteriores al
parto. La interesada podrá optar porque se le reduzca la licencia
anterior al parto, que en ningún caso podrá ser inferior a 30 días,
acumulándose los días reducidos al período posterior. En el caso de
parto múltiple se ampliará en 15 días corridos por cada alumbramiento
adicional. En el supuesto de que se adelante o difiera el parto, se
reconsiderará la fecha inicial de la licencia otorgada, de acuerdo a
cuándo aquél se haya producido efectivamente. Los días previos a la
fecha a partir de la cual le hubiera correspondido licencia por
maternidad, se computarán como períodos que se conceden por afecciones o
problemas de salud de corto o largo tratamiento. Este mismo criterio se
aplicará en los casos de los hijos nacidos muertos.

Art. 107º. Permiso diario por lactancia: Toda trabajadora madre de
lactante podrá disponer de dos permisos de media hora durante su jornada
laboral para amamantar a su hijo, por un período máximo de doscientos
cuarenta días posteriores a la fecha del cese de la licencia por
maternidad, salvo casos excepcionales certificados en que podrá
extenderse hasta un total de un año. La trabajadora podrá optar por
acumular las dos medias horas al principio o al final de la jornada, o
tomarlas por separado.

Art. 108º: Todo trabajador no docente, que tenga a su cargo un hijo con
capacidades diferentes podrá disponer de dos permisos de media hora
durante su jornada laboral, según elija, para poder atenderlo de manera
adecuada.

Excepcionalmente, y estando debidamente acreditada la necesidad de mayor
atención, podrá extenderse dicho permiso diario, media hora más por
jornada laboral. Siempre se exigirá la presentación de las
certificaciones médicas correspondientes.

Se considera que la persona tiene capacidades diferentes cuando padezca
una alteración funcional permanente o prolongada, física o mental, que
en relación a su edad y medio social implique desventajas considerables
para su integración familiar, social, educacional o laboral.

Las Paritarias particulares podrán ampliar esta concesión analizando
pormenorizadamente cada caso en la medida que se presente en cada
Institución Universitaria nacional.

En caso que ambos progenitores trabajen en la misma Institución
Universitaria nacional, uno sólo de ellos podrá acceder al beneficio.

Art. 109º. Opción a favor de la trabajadora. Estado de excedencia: La
trabajadora con más de un año de antigüedad en la Institución
Universitaria que tuviera un hijo, luego de gozar de la licencia por
maternidad, podrá optar entre las siguientes alternativas:

a) Continuar con su trabajo en la Institución Universitaria en las
mismas condiciones en que lo venía haciendo.

b) Quedar en situación de excedencia sin goce de sueldo por un período
de hasta seis meses.

c) Solicitar la resolución de la relación de empleo, con derecho a
percibir una compensación equivalente al 50\% del mejor salario de los
últimos 10 años por cada año de antigüedad en la Institución
Universitaria.

Para hacer uso de los derechos acordados en los incisos b) y c) deberá
solicitarlo en forma expresa y por escrito, y en el último caso hacerlo
dentro de los treinta días corridos de su reincorporación.

Art. 110º. Adopción: En caso de adopción la trabajadora tendrá derecho a
una licencia con goce de haberes de 45 días corridos a partir de la
fecha en que se otorgue la tenencia con fines de adopción; igual
beneficio tendrá el trabajador que adopte como único padre al menor.
Transcurrido ese período, la situación del trabajador adoptante quedará
asimilada a la de la maternidad. Para tener derecho a este beneficio
deberá acreditar la decisión judicial respectiva.

Art. 111º. De los feriados obligatorios y días no Laborales: Se regirán
de acuerdo a lo establecido en la legislación vigente.

Art. 112º. Día del trabajador no docente: Será asueto para el personal
no docente el día 26 de noviembre de cada año, DIA DEL TRABAJADOR NO
DOCENTE. Las Paritaria Particulares acordarán la forma en que se cumplan
las guardias mínimas que permitan atender todos los servicios esenciales
de la Institución Universitaria.

Art. 113º: El uso de las licencias disponibles reguladas en los
artículos 100º, 101º y 102º será considerado en la evaluación anual del
agente.

\subsection{TÍTULO 7}\label{tuxedtulo-7}

\hyperdef{}{salud-e-higiene}{\subsubsection{SALUD E
HIGIENE}\label{salud-e-higiene}}

Art. 114º: Las Instituciones Universitarias nacionales deben hacer
observar las pautas y limitaciones al trabajo establecido en leyes,
decretos y reglamentaciones y adoptar las medidas según el tipo de
trabajo, y que la experiencia y la técnica hagan necesarias para tutelar
la integridad psico-física y la dignidad de los trabajadores, debiendo
evitar los efectos perniciosos de las tareas penosas, riesgosas o
determinantes de vejez o agotamiento prematuro, así como también los
derivados de ambientes insalubres o ruidosos.

A esos efectos llevarán a cabo las siguientes tareas:

a) crearán servicios de seguridad e higiene de trabajo de carácter
preventivo y correctivo acorde a las especificaciones dadas en el marco
de las leyes vigentes con la participación gremial correspondiente.

b) mantendrán en un buen estado de conservación, utilización y
mantenimiento de los equipos, instalaciones, oficinas y todos los útiles
y herramientas de trabajo.

c) mantendrán en un buen estado de utilización y funcionamiento las
instalaciones eléctricas, sanitarias y de agua potable.

d) evitarán la acumulación de desechos, residuos y elementos que
constituyan riesgos para la salud o puedan producir accidentes,
efectuando en forma periódica la limpieza y las desinfecciones
pertinentes.

e) adoptarán medidas para eliminar y/o aislar los ruidos y/o las
vibraciones perjudiciales para la salud de los trabajadores, brindando
elementos de protección adecuados si aquello resulta técnica y
económicamente viable.

f) instalarán equipos para afrontar los riesgos en casos de incendio y
los demás siniestros que pudieran ocurrir.

g) deberán promover la capacitación del personal en materia de higiene y
seguridad de trabajo, particularmente en lo referido a la prevención de
los riesgos específicos de las tareas asignadas.

h) adoptarán medidas de resguardo y seguridad frente al efecto de las
sustancias peligrosas que se encuentren en el ámbito de la institución
universitaria.

i) desarrollarán un plan de evacuación y roles en caso de emergencia
(incendio y otros).

Art. 115º: El personal queda comprometido a:

a) cumplir las normas de seguridad e higiene referentes a las
obligaciones de uso, conservación y cuidado de equipos de protección
personal y de los propios de las maquinarias, operaciones y procesos de
trabajo.

b) conocer y cumplir debidamente las normas de seguridad de la
Institución Universitaria, con un criterio de colaboración y seguridad
por ambas partes.

c) someterse a los exámenes médicos preventivos y periódicos que indique
la Institución Universitaria. Esta invitará a la entidad gremial para
que disponga la presencia de sus facultativos, si lo estima conveniente.

d) cuidar la conservación de los carteles y avisos que señalan medidas
de seguridad e higiene, y observar sus prescripciones.

e) colaborar en la elaboración del Programa de formación y educación en
materia de higiene y seguridad, y asistir a los cursos que se dicten
durante la jornada de trabajo.

f) denunciar, conforme las normas legales vigentes, los accidentes o
enfermedades laborales.

Art. 116º: Se constituye la Comisión de ``Condiciones y Medio Ambiente
del Trabajo (CCyMAT)'', que estará integrada por dos expertos por cada
sector, debiendo contar al menos con dos especialistas en medicina
laboral. Se financiará con el aporte conjunto de las partes,
requiriéndose también al MTSS y a la OIT (Departamento CyMAT) apoyo
técnico. Las resoluciones de esta Comisión serán de aplicación
obligatoria para las Instituciones Universitarias nacionales y sus
agentes.

Art. 117º: La CCyMAT tendrá por funciones:

a) confeccionar un manual de instrucciones preventivas para todo el
personal dependiente que tienda a evitar enfermedades profesionales y
los accidentes de trabajo.

b) fiscalizar el cumplimiento de lo dispuesto en el presente título, en
todas las Instituciones Universitarias nacionales, elevando dictámenes
trimestrales a la Comisión negociadora de nivel general.

c) establecer las medidas necesarias para subsanar o atenuar la
situación planteada por la tarea insalubre o riesgosa. Para ello contará
con los informes producidos por la Aseguradora de Riesgo de Trabajo o
las comisiones de salubridad e higiene que pudieran haberse constituido
en el marco de la Disposición DNHST Nº 729/88 y concordantes.

Art. 118º: Deberá privilegiarse la implementación de medidas que
resguarden la salud del trabajador y minimicen los riesgos en el
trabajo, y sólo se recurrirá a la compensación pecuniaria en los casos
en que sea ineludible la exposición perjudicial. Esta última será la que
determine la Superintendencia de Riesgos del Trabajo o la entidad que en
el futuro la reemplace.

\subsection{TÍTULO 8}\label{tuxedtulo-8}

\hyperdef{}{capacitacion}{\subsubsection{CAPACITACION}\label{capacitacion}}

Art. 119º: Las Instituciones Universitarias nacionales deberán ofrecer a
sus trabajadores cursos de capacitación permanente, que posibiliten su
crecimiento personal y el mejor desempeño de sus funciones. Se
desarrollarán con criterios de pertinencia respecto de las funciones que
desempeñen o puedan desempeñar, sin que esto entorpezca la carrera
administrativa.

Art. 120º: Tendrá por objetivos generales:

a) proporcionar competencias específicas para afrontar los nuevos
desafíos laborales;

b) potenciar habilidades, conocimientos y experiencia;

c) reducir los requerimientos de supervisión y otorgar mayor autonomía
decisional;

d) mejorar las oportunidades de promoción y progreso, propendiendo al
desarrollo pleno de su carrera dentro de la institución;

e) proporcionar mayor seguridad, satisfacción en el trabajo y
realización personal.

Estará orientada a:

1. elevar los niveles de productividad, con un mejor uso de los recursos
disponibles;

2. mejorar la gestión para poder asumir las rápidas transformaciones
características de nuestro tiempo;

3. generar las condiciones para que cada uno de los miembros de la
organización contribuyan con sus capacidades y desempeños a un mejor
logro de los objetivos institucionales.

Art. 121º: La Institución Universitaria establecerá planes de
capacitación consensuados en las comisiones paritarias particulares.

Art. 122º: El desarrollo de la carrera individual es responsabilidad de
cada uno de los trabajadores, quienes deberán realizar los esfuerzos
necesarios para su progreso personal.

Art. 123º: La capacitación general que atienda a completar la educación
general básica obligatoria de los trabajadores será gratuita, y podrá
cumplirse dentro o fuera del horario de trabajo.

Art. 124º: Cuando la capacitación tenga que ver con procesos o
conocimientos a los que el trabajador deba acceder para adecuarse a la
modernización de la tarea o a la aplicación a áreas creadas a posteriori
de su incorporación en el trabajo, los cursos serán gratuitos y en
horario de trabajo.

Art. 125º: La Institución Universitaria se compromete a otorgar la
posibilidad de ingreso a los trabajadoras no docentes al circuito de
formación y capacitación a la totalidad de los oficios y especialidades
que se desarrollen en la Institución Universitaria.

Art. 126º: Si los cursos ofrecidos no comprendidos en los artículos
anteriores estuviesen arancelados, se implementarán los acuerdos
necesarios para permitir el acceso equitativo a todo trabajador
interesado en realizarlo.

Art. 127º: En todo proceso de concurso para la asignación de una
categoría superior, se tendrá especialmente en cuenta la capacitación
acreditada con las actividades formativas institucionalizadas.

Art. 128º: Las Instituciones Universitarias nacionales podrán reconocer
incentivos pecuniarios a la capacitación del personal, cuando así fuera
acordado por las partes.

Art. 129º: Créase la Comisión Asesora de Capacitación, que tendrá las
siguientes funciones:

a) Asesorar en la formulación de políticas y programas de capacitación,
convergentes con los criterios generales, realizando observaciones o
sugerencias.

b) Coordinar las actividades de las Instituciones Universitarias
nacionales de manera de regionalizar los procesos y generar acciones
cooperativas para el mejor aprovechamiento de los recursos humanos que
puedan emplearse en la capacitación.

c) Gestionar fondos ante los organismos competentes para llevar a cabo
programas de capacitación.

La Comisión Asesora de Capacitación estará integrada por cuatro
miembros, dos a propuesta de cada una de las partes, y actuará con las
instrucciones y bajo la dirección de la Comisión Negociadora de Nivel
General.

Art. 130: En los casos en que haya una asignación de fondos específica
para estos fines, la Comisión Asesora de Capacitación auditará su uso.

\subsection{TÍTULO 9}\label{tuxedtulo-9}

\subsubsection{EVALUACION DE
DESEMPEÑO}\label{evaluacion-de-desempeuxf1o}

Art. 131º: Se entiende por evaluación de desempeño, la realizada acerca
de competencias, aptitudes y actitudes del trabajador, y del logro de
objetivos o resultados en sus funciones.

Art. 132º: La evaluación de desempeño deberá Contribuir a estimular el
compromiso del trabajador con el rendimiento laboral y la mejora
organizacional, su desarrollo y capacitación, la profesionalidad de su
gestión y la ponderación de la idoneidad relativa.

Art. 133º: La evaluación de desempeño se hará en Forma regular,
anualmente, y será tomada en cuenta para elaborar políticas de recursos
humanos, capacitación e incentivos y como antecedente en las promociones
y los concursos.

Art. 134º: Cada agente será evaluado por un órgano de evaluación que en
cualquier caso deberá integrarlo el jefe inmediato superior. A tal
efecto se confeccionará un formulario, donde se registren calificaciones
de 1 a 10 respecto de los factores que se mencionan a continuación:

a) nivel de presentismo y puntualidad

b) responsabilidad

c) conocimiento del área donde se desempeña

d) iniciativa

e) eficiencia, eficacia y creatividad

f) espíritu de colaboración

g) ánimo de superación

h) corrección personal

i) sanciones disciplinarías merecidas durante el período evaluado.

Además, para el tramo superior, se evaluará:

j) capacidad de planificación y organización

k) capacidad de conducción y liderazgo

I) objetividad y compromiso en el manejo del área

m) aptitud para calificar.

Una vez evaluados los ítems antes mencionados, se consignará su
promedio. El puntaje mínimo para considerar la evaluación como positiva
será mayor a cinco.

La evaluación del desempeño del agente podrá complementarse con las
ponderaciones de otros actores vinculados con su gestión. En caso que un
agente con personal a su cargo (el evaluador) se desvincule del servicio
deberá dejar un informe sobre el desempeño de los agentes que de él
dependan.

Art. 135º: La evaluación de desempeño será notificada al agente, dentro
de los cinco días de producida. En caso de disconformidad, el agente
podrá, dentro de los cinco días de notificado, interponer recurso ente
la Junta Superior de Calificación.

Art. 136º: La Junta Superior de Calificación se constituirá a nivel de
Institución Universitaria y estará conformada por al menos cinco (5)
miembros. Su integración será resuelta en paritarias particulares.

Art. 137º: Las decisiones de la Junta se adoptarán por simple mayoría de
votos de los miembros presentes en la sesión.

El quórum para sesionar será de tres (3) miembros. En las reuniones de
la Junta Superior de Calificación podrá participar el gremio local, a
través de un representante y en calidad de vendedor.

Art. 138º: La Junta Superior de Calificación deberá expedirse dentro de
los quince (15) días hábiles de la fecha de interposición del recurso.
La notificación de esta resolución deberá formalizarse dentro de los
cinco (5) días hábiles siguientes. Este pronunciamiento cierra la vía
administrativa.

Todas las actuaciones se agregarán al legajo del agente.

Art. 139º: Cada tres años se llevará a cabo un análisis cuantitativo y
cualitativo de las evaluaciones producidas por cada agente evaluador,
para determinar su objetividad y eficiencia. Estará a cargo de
funcionarios de la gestión, y su resultado se hará constar en el legajo
respectivo.

\subsection{TÍTULO 10}\label{tuxedtulo-10}

\hyperdef{}{regimen-disciplinario}{\subsubsection{REGIMEN
DISCIPLINARIO}\label{regimen-disciplinario}}

Art. 140º: Los trabajadores no docentes de las Instituciones
Universitarias nacionales se encontrarán sujetos a las siguientes
medidas disciplinarias:

a) apercibimiento

b) suspensión de hasta treinta (30) días;

c) cesantía;

d) exoneración.

Estas sanciones se aplicarán sin perjuicio de las responsabilidades
civiles y penales que fijen las leyes vigentes.

Las suspensiones se harán efectivas sin prestación de servicios ni
percepción de haberes, en la forma y los términos que determine la
reglamentación.

El cómputo de las sanciones se hará por cada trasgresión en forma
independiente y acumulativa, pudiendo ser aplicadas en un solo acto.

En ningún caso el trabajador podrá ser sancionado más de una vez por la
misma causa.

Toda sanción se graduará teniendo en cuenta la gravedad de la falta, los
antecedentes del agente y los perjuicios causados.

Art. 141º: Son causales de la sanción de apercibimiento:

a) incumplimiento injustificado del horario de trabajo, de acuerdo a la
reglamentación que se acuerde en el ámbito de la Comisión Negociadora de
Nivel General;

b) falta de respeto leve a miembros de la comunidad universitaria o al
público;

c) negligencia menor en el cumplimiento de las funciones,

Art. 142º: Son causales de la sanción de suspensión:

a) inasistencias injustificadas que no excedan los doce (12) días
discontinuos de servicio, en el lapso de doce meses inmediatos
anteriores a la primera;

b) falta de respeto grave a miembros de la comunidad universitaria o al
público; excepto que por su magnitud implique una sanción mayor;

c) incumplimiento deliberado y no grave de las obligaciones y
prohibiciones del régimen de empleo.

d) reincidencia, reiteración o agravación de las causales de
apercibimiento del artículo anterior.

Art. 143º: Son causales de cesantía:

a) inasistencias injustificadas que excedan los doce (12) días
discontinuos de servicio en el lapso de los doce meses inmediatos
posteriores a la primera;

b) abandono del servicio, que se configurará cuando medien seis (6) o
más inasistencias injustificadas consecutivas del agente, y se haya
cursado intimación fehaciente a retomar el servicio, emanada de
autoridad competente, sin que ello se hubiera producido dentro de los
dos días subsiguientes a la intimación;

c) falta de respeto a miembros de la comunidad universitaria o al
público cuya magnitud afecte de tal forma a las personas o a la
institución universitaria que desaconseje la continuidad en el empleo,
lo que deberá estar suficientemente fundamentado;

d) incumplimiento deliberado y grave de las obligaciones y prohibiciones
del régimen de empleo.

e) acumulación de treinta días de suspensión en los doce meses
inmediatos anteriores;

f) quedar el agente incurso en alguna de las situaciones previstas en el
art. 21º incs. a) o d); en este último caso sólo cuando la sanción
sobreviniente sea de cesantía.

Art. 144º: Serán causales de exoneración:

a) falta grave que perjudique material o moralmente a la Institución
Universitaria;

b) condena firme cuya pena principal o accesoria sea la inhabilitación
absoluta o especial para ejercer cargos públicos.

c) sentencia condenatoria firme por delito contra la Administración
Pública Nacional, Provincial o Municipal o contra cualquier Institución
Universitaria Nacional,

d) quedar el agente incurso en alguna de las situaciones previstas en el
art. 21º incs. b), c), d) ---en caso de que la sanción hubiera sido de
exoneración--- o e).

Art. 145º. Procedimiento: A los fines de la aplicación de estas
sanciones, se requerirá la instrucción de un sumario previo, conforme el
procedimiento que se establezca en la reglamentación, el cual deberá
garantizar al imputado el derecho a su defensa.

El sumario deberá estar concluido en el lapso de seis (6) meses, plazo
que podrá ser prorrogado por causa fundada.

Quedan exceptuados de la exigencia del sumario previo los casos
previstos en los artículos 141º, 142º inc. a), 143º inc. a), b), e) y f)
y 144º inc. b), (c) y d) en los que la sanción la resolverá directamente
la autoriad sobre la base de la prueba documental expedida.

Por vía reglamentaria se determinarán las autoridades con atribuciones
para aplicar las sanciones, como así también el procedimiento de
investigación aplicable.

Art. 146º. Suspensión preventiva: El personal sumariado podrá ser
preventivamente suspendido o trasladado con carácter transitorio por la
autoridad competente, cuando su alejamiento sea necesario para el
esclarecimiento de los hechos investigados o cuando su permanencia en
funciones fuere inconveniente, en la forma y términos que determine la
reglamentación. En caso de haberse aplicado suspensión preventiva y que
los resultados del sumario no sugieran sanciones o no fueran privativas
de haberes, éstos les serán liquidados como corresponda.

Art. 147º: La acción disciplinaria correspondiente a los artículos 141º
y 142º prescribirá a los seis (6) meses de cometida la falta, o desde
que la Institución Universitaria tome conocimiento de ésta, siendo de un
(1) año la prescripción en los restantes casos. En cualquier caso la
iniciación del sumario, cuando corresponda, interrumpirá la
prescripción.

Art. 148º: La sustanciación de los sumarios por hechos que puedan
configurar delitos y la imposición de las sanciones pertinentes en el
orden administrativo son independientes de la causa criminal, excepto en
aquellos casos en que da la sentencia definitiva surja la configuración
de una causal más grave que la sancionada; en tal supuesto se podrá
sustituir la medida aplicada por otra de mayor gravedad. El
sobreseimiento o la absolución dictados en la causa criminal, no
afectarán la sanción dispuesta en el orden administrativo.

\subsection{TÍTULO 11}\label{tuxedtulo-11}

\subsubsection{DISPOSICIONES GENERALES}\label{disposiciones-generales}

Art. 149º: En el marco del presente convenio, las Instituciones
Universitarias nacionales asignarán un área de responsabilidad
específica para su implementación y ejecución y de aquellos acuerdos que
se articulen a nivel particular. Tendrá como misiones y funciones las
de:

a) coordinar, gestionar y ejecutar políticas tendientes a poner en
marcha los acuerdos,

b) mantener un sistema de información interconectado entre sus
dependencias y con la entidad que represente la unificación de la parte
empleadora en los siguientes temas:

1.- Administración del trabajo (régimen sindical, inspección y control
de normas laborales al interior del sistema, higiene y salubridad,
accidentes de trabajo y atención de enfermedades, y normas relativas a
la protección del trabajador, negociación y tratamiento de conflictos)

2.- Capacitación

3.- Políticas de empleo (cambio de funciones, retiros).

Art. 150º: La Comisión Paritaria Particular verificará la adecuación de
la aplicación de los acuerdos convencionales a que se hace referencia en
el artículo anterior y de las excepciones que se dispongan en virtud de
lo establecido en este convenio.

Art. 151º: A los efectos de garantizar la adecuada articulación y
coordinación de la negociación colectiva particular, las Comisiones
negociadoras de ese nivel enviarán a la Comisión Negociadora de Nivel
General, previo a su homologación, copia de las actas acuerdo que se
concreten en sus respectivos ámbitos.

Art. 152º: La Comisión de Interpretación de Convenios y Solución de
Conflictos, tendrá las siguientes funciones:

a) interpretar el Convenio Colectivo de los trabajadores no docentes de
las Instituciones Universitarias nacionales a pedido de cualquiera de
las partes, conforme lo establece el art. 14º del Decreto 1007/95.

b) interpretar los acuerdos particulares que las propias Comisiones
Negociadoras de Nivel particular le sometan.

c) resolver las diferencias que puedan originarse entre las partes, ya
sea con motivo de la aplicación del Convenio Colectivo del sector o por
cualquier otra causa que esté vinculada con la relación laboral.

Esta Comisión estará compuesta por cuatro miembros, dos a propuesta de
cada una de las partes, los que durarán en sus funciones y serán
reemplazados en la forma y con las modalidades que resuelva la parte que
haya conferido la representación. Seguirá en su funcionamiento los
siguientes pasos:

1) las situaciones requeridas deberán ser consideradas por la Comisión
en un plazo máximo de cinco días hábiles de presentada la solicitud por
ambas o cualquiera de las partes. Para el caso en que la presentación no
haya sido conjunta, se dará traslado a la otra, dentro de los cinco
días, y por un plazo igual.

2) con ambos elementos la Comisión deberá expedirse dentro de los cinco
días de vencido el último plazo del inciso anterior.

3) mientras se estén substanciando las causas en cuestión, las partes se
abstendrán de realizar medidas de cualquier tipo que afecten el normal
funcionamiento del servicio, dejándose aclarado que durante ese lapso
quedarán en suspenso las medidas de carácter colectivo que hayan sido
adoptadas con anterioridad por cualquiera de las partes.

4) agotada la instancia prevista sin haberse arribado a una solución,
cualquiera de las partes podrá presentarse ante la autoridad laboral de
aplicación, solicitando la apertura de la negociación correspondiente.

5) quedará sujeto a análisis y resolución de esta Comisión determinar el
alcance de las modificaciones legislativas que se puedan producir con
relación aspectos contemplados en el presente convenio.

6) Finalmente, en cuanto a lo que esta Comisión interprete con respecto
al Convenio Colectivo (puntos a y b de sus funciones), sus facultades no
podrán excluir de ninguna manera a las judiciales.

El presente reglamento cuenta con la aprobación prevista por el art. 14º
primer párrafo del Decreto Nº 1007/95.

Art. 153º: Normas de aplicación supletoria: Sin perjuicio de lo
establecido en el presente convenio, para todo lo no previsto tanto en
la negociación colectiva general como en las particulares, se estará a
las normas establecidas en cada Institución Universitaria para el
personal no docente, conforme las atribuciones del artículo 75 inciso 19
de la Constitución Nacional. En lo referente a la situación del personal
no docente no rigen las disposiciones relativas al personal de la
Administración Pública centralizada (Ley 25.164) ni en forma supletoria,
por tratarse de una relación de empleo totalmente autorregulada conforme
lo dispuesto en la Ley 24.447 art. 19; Decreto 1007/95, Ley 24.521 de
Educación Superior, art. 59 inc b); queda, asimismo, expresamente
excluido de los casos que siguen bajo los alcances de la derogada ley
22.140 y todos sus decretos reglamentarios y anexos.

En forma provisoria, y hasta tanto se encuentre plenamente vigente lo
referente a la estructura salarial y escalafonaria propuesta en el
presente convenio, se mantienen como supletorias, en todo lo atinente y
que no hayan sido ya establecidas en el presente, las normas del Decreto
2213/87.

En todos los casos deberá prevalecer la norma más favorable al
trabajador, así sea de la Institución Universitaria nacional donde
presta servicios. En el caso en que la aplicación de esta disposición
generara una diferencia de criterios se procederá en la forma
establecida en el artículo 152º del presente Convenio Colectivo.

Art. 154º: Todas las modificaciones escalafonarias y remunerativas
previstas o derivadas de lo acordado en el presente Convenio Colectivo
sólo resultarán aplicables para el caso de existir una asignación
presupuestaria específica para dichos rubros, otorgada por el Ministerio
de Educación de la Nación, conforme lo convenido en el Acta Acuerdo
firmada entre el CIN, FATUN y el Ministerio de Educación, el día 4 de
noviembre de 2004.

\begin{figure}[htbp]
\centering
\includegraphics{http://www.fatun.org.ar/convenio_archivos/dto366-5-4-2006-1.jpg}
\caption{}
\end{figure}

\begin{figure}[htbp]
\centering
\includegraphics{http://www.fatun.org.ar/convenio_archivos/dto366-5-4-2006-2.jpg}
\caption{}
\end{figure}

\begin{figure}[htbp]
\centering
\includegraphics{http://www.fatun.org.ar/convenio_archivos/dto366-5-4-2006-3.jpg}
\caption{}
\end{figure}

\begin{figure}[htbp]
\centering
\includegraphics{http://www.fatun.org.ar/convenio_archivos/dto366-5-4-2006-4.jpg}
\caption{}
\end{figure}

\begin{figure}[htbp]
\centering
\includegraphics{http://www.fatun.org.ar/convenio_archivos/dto366-5-4-2006-5.jpg}
\caption{}
\end{figure}

\subsection{PARTE IV: AGRUPAMIENTO
ASISTENCIAL}\label{parte-iv-agrupamiento-asistencial}

\subsubsection{Subgrupo A: Profesional.}\label{subgrupo-a-profesional.}

Se aplicará, en lo pertinente, las definiciones realizadas para el
Agrupamiento Técnico --- Profesional, A.--- Profesional.

\subsubsection{Subgrupo B}\label{subgrupo-b}

Se aplicará, en lo pertinente, las definiciones realizadas para el
Agrupamiento Técnico --- Profesional, según el caso.

\subsubsection{Subgrupo C}\label{subgrupo-c}

Se aplicará, en lo pertinente, las definiciones realizadas para el
Agrupamiento Administrativo.

\subsubsection{Subgrupo D}\label{subgrupo-d}

Se aplicará, en lo pertinente, las definiciones realizadas para el
Agrupamiento Mantenimiento, Producción y Servicios.

\subsection{-FE DE ERRATAS---}\label{fe-de-erratas}

\textbf{Decreto 366/2006}

En la edición del 5 de abril de 2006, en la que se publicó el mencionado
Decreto, se deslizaron errores de compaginación en las Partes II y III
del Tipificador de Funciones, que integra el Convenio Colectivo anexo a
dicha norma, motivo por el cual se transcriben a continuación en forma
correcta dichas Partes

\begin{figure}[htbp]
\centering
\includegraphics{http://www.fatun.org.ar/convenio_archivos/dto366-5-4-2006-3.jpg}
\caption{}
\end{figure}

\begin{figure}[htbp]
\centering
\includegraphics{http://www.fatun.org.ar/convenio_archivos/dto366-5-4-2006-4.jpg}
\caption{}
\end{figure}

\begin{figure}[htbp]
\centering
\includegraphics{http://www.fatun.org.ar/convenio_archivos/dto366-5-4-2006-5.jpg}
\caption{}
\end{figure}

\end{document}
